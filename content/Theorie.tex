\chapter{Theorie}

\section{Einleitung}

In der Physik und theoretischen Chemie hat sich die MCTDH - Methode als effizienter Algorithmus zur L"osung der zeitabh"angige Schr"odingergleichung etabliert.
Die Wellenfunktion wird als ein Satz von zeitabh"angigen Basisfunktionen dargestellt.
Diese Basisfunktionen, die auch Einteilchenfunktionen genannt werden, k"onnen so gew"ahlt werden, dass die Einteilchenfunktionen sich aus mehrdimensionale Wellenfunktionen zusammensetzen, die wiederum als MCTDH - Wellenfunktion entwickelt werden.
Durch die rekursive Anwendung der MCTDH - Methode auf seine Einteilchenfunktionen wird ein mehrlagiges MCTDH entwickelt, dass quantumdynamische Rechnungen von Systemen mit bis zu 1000 Freiheitsgraden ermo"glicht. [Manthe, 2008 multilayer MCTDH approach]
Im Unterschied zu anderen quantumdynamische Methoden resultiert die Effizienz des MCTDH aus der zweischichtigen Darstellung dieser Methode.
Anstelle die Wellenfunktion in einer zeitunab"angigen Basis zu entwickeln und die Zeitentwicklung durch zeitabh"angige Entwicklungskoeffizienten zu beschreiben, 



 \begin{equation}
 F_{s}^{2}=|\Braket{\psi_{f}}{\psi_{i}}|^{2}
 \label{eq:franck1}
 \end{equation}
a
 \subsection{Absorption und Emission}
 \begin{figure}
\centering
\includegraphics[width=0.7\linewidth]{figures/Franck}
\caption{Schematische Darstellung zweier Potentialkurven des elektronischen Grundzustandes und eines angeregten Zustandes mit den jeweiligen Vibrationszust�nden. Absorption (blauer Pfeil) erfolgen bei gleicher Kerngeometrie.}
\label{fig:Franck}
\end{figure}
 \begin{equation}
k_{ET} \propto E^{2}\propto \frac{\mu_{D}^{2}\mu_{A}^{2}}{R_{DA}^{6}}
\label{eq:forster2}
 \end{equation}
  Gleichung \ref{eq:forster2} spiegelt die Abstandsabh�ngigkeit der Energietransferrate durch Coulombwechselwirkung wider.\cite{turro:1991}
