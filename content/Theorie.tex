\chapter{MCTDH Theorie}

\section{Einleitung}

In der theoretischen Chemie wird die MCTDH - Methode verwendet, um quantendynamische Rechnungen effizient zu berechnen.
Um die zeitabh"angige Schr"odingergleichung (SGL) eines mehrdimensionalen Systems zu l"osen, muss zun"achst die Wellenfunktion definiert werden.
Diese wird in der Standardmethode durch das Produkt von mehrdimensionalen zeitabh"angigen Basisfunktionen dargestellt. Diese Basisfunktionen werden in
einer eindimensionalen zeitunabh"angigen Basis mit den jeweiligen zeitabh"angigen Koeffizienten entwickelt.
F"ur jeden Freiheitsgrad $f$ des Systems ergeben sich $N$ zeitunabh"angige Basisfunktionen. Somit w"achst die Anzahl der Entwicklungskoeffizienten wie $N^{f}$ und
die Standardmethode skaliert exponentielle, sodass nur kleinere Systeme berechenbar sind. [meyer rev 2011]
  \\ Im Unterschied zu anderen quantendynamische Methoden resultiert die Effizienz des MCTDH aus seiner Doppellayerstruktur.
Anstelle die Wellenfunktion in einer zeitunab"angigen Basis zu entwickeln und die Zeitentwicklung durch zeitabh"angige Entwicklungskoeffizienten zu beschreiben,
wird in der MCTDH - Methode die Wellenfunktion als ein Satz von zeitabh"angigen Basisfunktionen dargestellt.
Diese zeitabh"angigen Basisfunktionen werden Einteilchenfunktionen (SPF) genannt und in einer primitiven zeitunabh"angigen Basis dargestellt.
Die Doppellayerstruktur des MCTDHs resultiert aus zwei Entwicklungen mit jeweils zeitabh"angigen Entwicklungskoeffiziente:
Zum einen stellen die Entwicklungskoeffizienten mit den SPFs die korrelierte Wellenfunktion dar und bilden den oberen MCTDH -
Layer und zum anderen k"onnen die SPFs durch die Entwicklungskoeffizienten in der primitiven zeitunabh"angigen Basis entwickelt werden. Diese Entwicklung bildet
den unteren Layer.[Manthe, 2008 multilayer MCTDH approach]
  \\ Die Anzahl der SPFs kann verglichen mit der primitiven Basis signifikant kleiner gew"ahlt werden.
Dennoch ist auch das MCTDH durch eine exponentielle Skalierung limitiert.
Um Korrelationseffekte beschreiben zu k"onnen, sind mindestens zwei SPFs pro Freiheitsgrad notwendig, sodass der numerische Aufwand mit der Anzahl der
Freiheitsgrade $f$ zu $2^f$ skaliert. Aufgrund dieser Skalierung ko"nnen Systeme mit maximal 12 - 14 korrelierten Koordinaten behandelt werden.
  \\ Zus"atzliche zu der Doppellayerstruktur k"onnen die Koordinaten in ,,logische`` und physikalische Koordinaten unterschieden werden und
  verschiedene physikalischen Koordinaten werden zu einzelne logische Koordinaten kombiniert. Die logischen Koordinaten werden Partikel genannt, sodass
  nicht die Anzahl der Freiheitsgrade der limiterende Faktor f"ur die modenkombinierte MCTDH - Rechnung ist,
  sondern die Anzahl der Partikel $p$. [Meyer, Cederbaum, 1996 und 1998]


\section{Layerstruktur der MCTDH - Wellenfunktion}

In der Standardmethode wird die Wellenfunktion in einer zeitunabh"angigen Basis $\mathcal{X}^{\kappa}_{j}(x_{\kappa})$ entwickelt:

 \begin{equation}
 \Psi(x_{1},..., x_{f}, t)=\sum^{N_{1}}_{j_{1}=1} ... \sum^{N_{f}}_{j_{f}=1} A^{1}_{j_{1}, ..., j_{f}}(t)\cdot \mathcal{X}^{(1)}_{j_{1}}(x_{1}) \cdot ... \cdot \mathcal{X}^{(f)}_{j_{f}}(x_{f})
 \label{Eq:Std_wave}
 \end{equation}

Im Unterschied zu Gleichung \ref{Eq:Std_wave} enth"alt die MCTDH - Wellenfunktion,

 \begin{equation}
 \Psi(x_{1},..., x_{f}, t)=\sum^{n_{1}}_{j_{1}=1} ... \sum^{n_{f}}_{j_{f}=1} A^{1}_{j_{1}, ..., j_{f}}(t)
 \cdot \phi^{1;1}_{j_{1}}(x_{1}, t) \cdot ... \cdot \phi^{1;f}_{j_{f}}(x_{f}, t)
 \label{Eq:mctdh_wave}
 \end{equation}

in den zeitabh"angigen SPFs $\phi^{\kappa}_{j}(x_{\kappa})$ dargestellt, die wiederum in der primitiven Basis $\mathcal{X}^{\kappa}_{j}(x_{\kappa})$ entwickelt werden:

\begin{equation}
 \phi^{1;\kappa}_{m} (x_{\kappa}, t)=\sum^{N_{\kappa}}_{j=1} A^{2;\kappa}_{m;j}(t) \cdot \mathcal{X}^{(\kappa)}_{j}(x_{1})
 \label{Eq:Std_wave}
 \end{equation}


Diese Basisfunktionen, die auch Einteilchenfunktionen genannt werden, k"onnen so gew"ahlt werden, dass die Einteilchenfunktionen sich aus mehrdimensionale
Wellenfunktionen zusammensetzen, die wiederum als MCTDH - Wellenfunktion entwickelt werden.
Durch die rekursive Anwendung der MCTDH - Methode auf seine Einteilchenfunktionen wird ein mehrlagiges MCTDH entwickelt, dass quantumdynamische Rechnungen
von Systemen mit bis zu 1000 Freiheitsgraden erm"oglicht. [Manthe, 2008 multilayer MCTDH approach]



 \subsection{Absorption und Emission}
