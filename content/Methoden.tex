\chapter{Technische Details}

Die Python-Schnittstelle f"ur MCTDH wurde in Cython erstellt. 
Die Programmiersprache Cython stellt eine Hybridprogrammiersprache dar,
die Python und C\textbackslash C++ kombiniert. Im Gegensatz zu Python
wird Cython kompiliert, wobei die Cythonsyntax der Pythonsyntax "ahnlich ist.
In Cython kann C++ Code verwendet werden, der beim Erstellen der Cython-Programme ebenfalls kompiliert wird.
Auf diese Weise wurden mehrere \textit{mctdh}-Klassen auf Basis von C++ Klassen erstellt.

Die graphisch Benutzeroberfl"ache (GUI) f"ur MCTDH-Rechnungen wurde in Python und Qt implementiert.
Der Zugriff auf die Qt-Bibliothek erfolgt "uber die Python-Bibliothek PyQt4. 
Das Design der einzelnen GUI-Fenster erfolgte in Qt-Designer. Die Informationen "uber 
die jeweiligen Fenster werden in \textit{ui}-Dateien gespeichert. Mit PyQt k"onnen diese Dateien eingelesen werden und aus den Dateien die entsprechenden 
Klassen erstellt und beliebig erweitert werden. So k"onnen
verschiedene Fenster in Qt-Designer erstellt werden und in PyQt miteinander verkn"upft werden.

Neben PyQt wurden die Python-Modul \textit{networkx} und \textit{matplotlib} verwendet.
Mithilfe von Klassen aus \textit{networkx} und \textit{matplotlib} konnten Baumdiagramme erstellt und gespei\-chert werden, die in der GUI zur 
Visualisierung des MCTDH-Baums verwendet wurden.

  




   
