\chapter{Einleitung}
\label{ch:einleitung}

Das zeitabh"angige Multikonfiguration-Hartree-Verfahren (MCTDH) 
\cite{MMC, MMC1} und seine Multilayererweiterung (ml-MCTDH)
\cite{WT3, Mreview2} 
sind effiziente Verfahren zur genauen Simulation mehrdimensionaler Quantendynamik,
die von mehreren Forschungsgruppen verwendet werden 
\cite{MCTDHex1, WestPNM, MCTDHex2, W1, MCTDHex4, MCTDHex5, MCTDHex6, MCTDHex7, MCTDHex8,
MCTDHex9, MCTDHex10, MCTDHex11, MCTDHex12, MCTDHex13, MCTDHex14, MCTDHex15, MCTDHex16, MCTDHex17,MCTDHex18}. 
Beispiele f"ur hochdimensionale 
Benchmark-Anwendungen sind die 21-dimen\-sio\-nalen Rechnungen, in denen die Tunnelaufspaltung des
Grundzustands \cite{CVM, HCVM, HaM1, MAMCTDH, HaM2, MAMCTDH2} und der angeregten \cite{HCVM, HaM1, MAMCTDH, HaM2, MAMCTDH2}
Schwingungszust"ande von Malonaldehyd erforscht wurden. 
Es wurden in 15-dimen\-sio\-nalen Rechnungen die Schwingungszust"ande von protonierten 
Wasserdi\-meren  \cite{H5O2+MCTDH, H5O2+MCTDH2, H5O2+MCTDH3, H5O2+MCTDH4, H5O2+MCTDH5} untersucht. 
Au"serdem wurde in 12-dimensionalen Rechnungen die thermischen 
Geschwindig\-keitskonstanten \cite{HM1, HM2, WWM, SM, vHNM,NvHM}, anfangszustandsausgew"ahlte Reaktionswahr\-scheinlichkeiten
\cite{SM02, SM04, WeM5, WeM6, WeM8} und die 
state-to-state Reaktionswahrscheinlichkeiten \cite{WeM7} f"ur die Reaktion von Methan mit Wasserstoff 
berechnet. In diesen Rechnungen wurden detaillierte \textit{ab initio} 
Potentialfl"achen verwendet. Signifikant h"ohere Dimensionen wurden in MCTDH-Rechungen mit 
Modelhamiltonoperatoren untersucht. So wurde in wegweisenden Rechnungen  
die nichtadibatischen Dynamiken von Pyrazin erforscht, in denen ein 24-modenschwingungsgekoppelter 
Hamiltonoperator \cite{WMC, WMC2, RWMC} verwendet wurde.
Multila\-yer-MCTDH Simulationen von typischen
physikalischen Modellen \cite{WT3, W1, WST, KCBWT, CTW2, WPHT} kondensierter Materie schlie"sen "ublich tausende Freiheitsgrade ein.
F"ur die Untersuchung eines Photodissoziationsmodels wurden in einem Wirt-Gast-Komplex 189-dimensionale 
Multila\-yer-MCTDH Rechnungen  durchgef"uhrt\cite{WBRSM}. 
%F"ur weitere Literatur, in der das MCTDH-Verfahren und seine Anwendungen diskutieren werden, siehe Refs. \cite{MCTDHreview, MCTDHreview2, HMreview1, MCTDHbook,Mreview2011, MCTDHreview3}.
F"ur weiterf"uhrende Literatur, in der das MCTDH-Verfahren und seine Anwendungen diskutiert werden, siehe Refs. \cite{MCTDHreview, MCTDHreview2, HMreview1, MCTDHbook,Mreview2011, MCTDHreview3}.

Das MCTDH-Programmpaket, welches zur Berechnung und Simulation der oben ge\-nann\-ten Systeme verwendet wird, bietet die M"oglichkeit,
Wellenfunktionen und Dichte\-matrizen hocheffizient zu propagieren, sowie Eigenzust"ande zu berechnen.
Bisher ist die Bedienung des vorliegenden MCTDH-Programmpakets selbst bei Standardaufgaben Spezialisten vorenthalten, da
ein tiefgreifendes Verst"andnis der Programmstruktur erforderlich ist. 
Im Rahmen dieser Masterarbeit wurden zwei wesentliche Verbesserungen unternommen: 
Es wurde eine Benutzeroberfl"ache (von englisch \textit{graphical user interface}, GUI) erstellt, die Wissenschaftlern
ohne Programmierkenntnisse die Bedienung des Programms bei Standardaufgaben erm"oglicht.
Des Weiteren wurde eine Python-Schnittstelle f"ur MCTDH entwickelt,
die es erlaubt, komplizierte Aufgaben mithilfe von Skripten zu bew"altigen.
Diese Schnittstelle erm"oglicht die Nutzung des MCTDH-Programmpakets, ohne 
%dass dessen Programmstruktur verstanden werden muss.
ein tiefgreifendes Verst"andnis der Programmstruktur vorrauszusetzen.
Python hat sich als einsteigerfreundliche Programmiersprache erwiesen. Vergleichbare Python-Schnitt\-stellen existieren auch f"ur viele andere bekannte und h"aufig genutzte 
numerische Programmpakete(TensorFlow \cite{TensorFlow}, SciPy \cite{SciPy} usw.).

Diese Arbeit ist wie folgt gegliedert: In Kapitel \ref{ch:theo} wird der Ansatz der MCTDH-Wellenfunk\-tion beschrieben. 
Es werden die Unterschiede zum Multilayer-MCTDH hervorgehoben. 
Die technische Beschreibung der GUI erfolgt in Kapitel \ref{cha:Tech}.
In Kapitel \ref{ch:Ergebnisse} werden die Python-Schnittstelle und die graphische Benutzeroberfl"ache vorgestellt.     
Schlie"slich wird in Kapitel \ref{cha:fazit} die Arbeit zusammengefasst und ein Ausblick gegeben.     