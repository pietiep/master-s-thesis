\chapter{Einleitung}
\label{ch:einleitung}

Das zeitabh"angige Multikonfiguration-Hartree-Verfahren (MCTDH) 
\cite{MMC, MMC1} und seine Multilayererweiterung (ml-MCTDH)
\cite{WT3, Mreview2} 
sind effiziente Verfahren zur genauen Simulation mehrdimensionaler Quantendynamik,
die von mehreren Forschungsgruppen verwendet werden 
\cite{MCTDHex1, WestPNM, MCTDHex2, W1, MCTDHex4, MCTDHex5, MCTDHex6, MCTDHex7, MCTDHex8,
MCTDHex9, MCTDHex10, MCTDHex11, MCTDHex12, MCTDHex13, MCTDHex14, MCTDHex15, MCTDHex16, MCTDHex17,MCTDHex18}. 
Beispiele f"ur hochdimensionale 
Benchmark-Anwendungen sind die 21-dimen\-sio\-nalen Rechnungen, in denen die Tunnelaufspaltung des
Grundzustands \cite{CVM, HCVM, HaM1, MAMCTDH, HaM2, MAMCTDH2} und der angeregten \cite{HCVM, HaM1, MAMCTDH, HaM2, MAMCTDH2}
Schwingungszust"ande von Malonaldehyd erforscht wurde. 
Es wurden in 15-dimen\-sio\-nalen Rechnungen die Schwingungszust"ande von protonierten 
Wasserdi\-meren  \cite{H5O2+MCTDH, H5O2+MCTDH2, H5O2+MCTDH3, H5O2+MCTDH4, H5O2+MCTDH5} untersucht. 
Au"serdem wurden in 12-dimensionalen Rechnungen die thermischen 
Geschwindig\-keitskonstanten \cite{HM1, HM2, WWM, SM, vHNM,NvHM}, anfangszustandsausgew"ahlte Reaktionswahr\-scheinlichkeiten
\cite{SM02, SM04, WeM5, WeM6, WeM8} und die 
state-to-state Reaktionswahrscheinlichkeiten \cite{WeM7} f"ur die Reaktion von Methan mit Wasserstoff 
untersucht. In diesen Rechnungen wurden detaillierte \textit{ab initio} 
Potentialfl"achen verwendet. Signifikant h"ohere Dimensionen wurden in MCTDH-Rechungen mit 
Modelhamiltonoperatoren untersucht. So wurde in wegweisenden Rechnungen  
die nichtadibatischen Dynamiken von Pyrazin erforscht, in denen ein 24-Moden 
schwingungsgekoppelter Hamiltonoperator \cite{WMC, WMC2, RWMC} verwendet wurde.
Multilayer-MCTDH-Simulationen von typischen
physikalischen Modellen \cite{WT3, W1, WST, KCBWT, CTW2, WPHT} kondensierter Materie schlie"sen "ublich tausende Freiheitsgrade ein.
F"ur die Untersuchung eines Photodissoziationsmodels wurden in einem Wirt-Gast-Komplex 189-dimensionale 
Multilayer-MCTDH-Rechnungen  durchgef"uhrt\cite{WBRSM}. 
%F"ur weitere Literatur, in der das MCTDH-Verfahren und seine Anwendungen diskutieren werden,
F"ur weiterf"uhrende Literatur, in der das MCTDH-Verfahren und seine Anwendungen diskutiert
werden, siehe Refs. \cite{MCTDHreview, MCTDHreview2, HMreview1, MCTDHbook,Mreview2011, MCTDHreview3}.

Es existieren mehrere Implementationen des MCTDH-Algorithmus' in unterschiedli\-chen Forschungsgruppen.
Ein viel genutztes Programmpaket ist das MCTDH-Programm der Universit"at Heidelberg\cite{Heidelberg}. Es
basiert auf einem haupts"achlich in FORTRAN77 verfassten numerischen Programmteil und einer
graphischen Benutzeroberfl"ache. In der Arbeitsgruppe der Universit"at Bielefeld wurde k"urzlich
ein neues MCTDH-Programm entwickelt, das auf einem objektorientiertem C++-Code basiert.
Es vereint einen effizienten numerischen Teil mit einer stark strukturierten Klassenarchitektur,
die den Code entwicklerfreundlich, sowie nutzerfreundlich zugleich macht. Allerdings erfordert
die Nutzung des Programms selbst bei vergleichsweise einfachen Aufgaben bislang
grundlegende Kenntnisse in C++.

Im Rahmen dieser Masterarbeit wurden zwei wesentliche Verbesserungen unternommen: 
Es wurde eine moderne Benutzeroberfl"ache (von englisch \textit{graphical user interface}, GUI) erstellt,
die Wissenschaftlern ohne Programmierkenntnisse die Bedienung des Programms bei Standardaufgaben erm"oglicht.
Des Weiteren wurde eine Python-Schnittstelle f"ur MCTDH entwickelt,
die es erlaubt, kompliziertere Aufgaben mithilfe von Skripten zu bew"altigen.
Python hat sich als einsteigerfreundliche Programmiersprache erwiesen. 
Vergleichbare Python-Schnittstellen existieren auch f"ur viele andere bekannte und h"aufig genutzte 
numerische Programmpakete(TensorFlow \cite{TensorFlow}, SciPy \cite{SciPy} u.w.).
Diese Schnittstelle erm"oglicht die Nutzung des MCTDH-Programmpakets, ohne dass 
dessen Programmstruktur verstanden werden muss.
 
Diese Arbeit ist wie folgt gegliedert: In Kapitel \ref{ch:theo} wird der Ansatz der MCTDH-Wellenfunk\-tion beschrieben. 
Es werden die Unterschiede zum Multilayer-MCTDH herausgestellt. Die Implementation der
Python-Schnittstelle und der MCTDH-GUI wird
in Kapitel \ref{cha:Implementation} erl"autert und deren Anwendungen demonstriert.
%In Kapitel 4 werden die Python-Schnittstelle und die graphischen Benutzeroberfl"ache beschrieben.     
Schlie"slich wird in Kapitel \ref{cha:fazit} ein Fazit gezogen und ein Ausblick gegeben.     
