\chapter{Einleitung}
\label{ch:einleitung}

Das zeitabh"angige Multikonfiguration-Hartree-Verfahren (MCTDH) 
\cite{MMC, MMC1} und seine Multilayererweiterung (ml-MCTDH)
\cite{WT3, Mreview2} 
sind effiziente Verfahren zur genauen Simulation mehrdimensionaler Quantendynamik,
die von mehreren Forschungsgruppen verwendet werden 
\cite{MCTDHex1, WestPNM, MCTDHex2, W1, MCTDHex4, MCTDHex5, MCTDHex6, MCTDHex7, MCTDHex8,
MCTDHex9, MCTDHex10, MCTDHex11, MCTDHex12, MCTDHex13, MCTDHex14, MCTDHex15, MCTDHex16, MCTDHex17,MCTDHex18}. 
Beispiele f"ur hochdimensionale 
Benchmark-Anwendungen sind die 21-dimen\-sio\-nalen Rechnungen, in denen die Tunnelaufspaltung des
Grundzustands \cite{CVM, HCVM, HaM1, MAMCTDH, HaM2, MAMCTDH2} und der angeregten \cite{HCVM, HaM1, MAMCTDH, HaM2, MAMCTDH2}
Schwingungszust"ande von Malonaldehyd erforscht wurde. 
Au"serdem wurden in 15-dimen\-sio\-nalen Rechnungen die Schwingungszust"ande von protonierten 
Wasserdimeren \cite{H5O2+MCTDH, H5O2+MCTDH2, H5O2+MCTDH3, H5O2+MCTDH4, H5O2+MCTDH5} untersucht. 
Zudem wurde in 12-dimensionalen Rechnungen die thermischen 
Geschwindig\-keitskonstanten \cite{HM1, HM2, WWM, SM, vHNM,NvHM}, anfangszustandsausgew"ahlte Reaktionswahrscheinlichkeiten
\cite{SM02, SM04, WeM5, WeM6, WeM8} und die 
state-to-state Reaktionswahrscheinlichkeiten \cite{WeM7} f"ur die Reaktion von Methan mit Wasserstoff 
untersucht. In diesen Rechnungen wurden detaillierte \textit{ab initio} 
Potentialfl"achen verwendet. Signifikant h"ohere Dimensionen wurden in MCTDH-Rechungen mit 
Modelhamiltonoperatoren untersucht. So wurde in wegweisenden Rechnungen  
die nichtadibatischen Dynamiken von Pyrazin erforscht, in denen ein 24-moden 
schwingungsgekoppelter Hamiltonoperator \cite{WMC, WMC2, RWMC} verwendet wurde.
Multilayer-MCTDH Simulationen von typischen
physikalischen Modellen \cite{WT3, W1, WST, KCBWT, CTW2, WPHT} zu kondensierter Materie schlie"sen "ublich tausende Freiheitsgrade ein.
F"ur die Untersuchung eines Photodissoziationsmodel wurden in einem Wirt-Gast-Komplex 189-
dimensionale ml-MCTDH Rechnungen  durchgef"uhrt\cite{WBRSM}. F"ur weitere Literatur zum MCTDH-Verfahren und
seine Anwendungen diskutieren, siehe Refs. \cite{MCTDHreview, MCTDHreview2, HMreview1, MCTDHbook,Mreview2011, MCTDHreview3}.
Bisher ist die Bedienung des vorliegenden MCTDH-Programmpakets Spezialisten vorenthalten und 
ein tiefgreifendes Verst"andnis des Codes erfordert. 


Im Rahmen dieser Masterarbeit wurden zwei wesentliche Verbesserungen unternommen: Zum eine wurde eine Eingabe-GUI implementiert, die Wissenschaftlern
ohne Programmierkenntnissen den Zugang zu eigenst"andigen MCTDH-Rechnungen erm"oglicht.
Des Weiteren wurde eine Python-Schnittstelle f"ur MCTDH entwickelt, sodass eine abstrakte Bedienung des MCTDH-Programmpakets in Python
realisiert werden konnte. Der MCTDH-Quellcode besteht aus mehreren Komponenten, die in den Programmiersprachen Fortran77 und C++ geschrieben sind und
 MCTDH-Routinen k"onnen nun in Python aufgerufen und benutzt werden ohne deren genaue Funktionsweise zu kennen.
 Python ist durch Zugriff auf verschiedene Python-Module wie TensorFlow \cite{TensorFlow} und SciPy\cite{SciPy} sehr verbreitet.
 
 Im Rahmen dieser Arbeit wurde kann "uber die GUI
 die MCTDH Basis eingelesen werden. Die Basis des MCTDHs wird durch eine Baumstruktur angegeben, die nun in Python bearbeitete werden kann.
 Des Weiteren wurde eine "ubersichtliche Projektverwaltung  von MCTDH-Rechnungen geschaffen.  
\\Diese Arbeit ist wie folgt gegliedert. In Kapitel 2 wird der Ansatz der MCTDH-Wellenfunk\-tion beschrieben. Es werden die Unterschiede zum ml-MCTDH herausgestellt.
In Kapitel 3 wird die Python-Schnittstelle und die graphischen Benutzeroberfl"ache beschrieben.     
Schlie"slich wird in Kapitel 4 eine Fazit gezogen und ein Ausblick gegeben.     