\chapter{Einleitung}
\label{ch:einleitung}

Das zeitabh"angige Multikonfiguration-Hartree-Verfahren (MCTDH) 
\cite{MMC, MMC1} und seine Multilayererweiterung (ml-MCTDH)
\cite{WT3, Mreview2} 
sind effiziente Verfahren f"ur genaue mehrdimensionale Quantendynamiksimulationen,
die von mehreren Forschungsgruppen verwendet werden. 
\cite{MCTDHex1, WestPNM, MCTDHex2, W1, MCTDHex4, MCTDHex5, MCTDHex6, MCTDHex7, MCTDHex8,
MCTDHex9, MCTDHex10, MCTDHex11, MCTDHex12, MCTDHex13, MCTDHex14, MCTDHex15, MCTDHex16, MCTDHex17,MCTDHex18} 
Beispiele f"ur hochdimensionale 
Benchmark-Anwendungen sind die 21-dimensionalen Rechnungen, in denen die Tunnelaufspaltung des
Grundzustands \cite{CVM, HCVM, HaM1, MAMCTDH, HaM2, MAMCTDH2} und der angeregten \cite{HCVM, HaM1, MAMCTDH, HaM2, MAMCTDH2}
Schwingungszust"ande von Malonaldehyd erforscht wird. 
Au"serdem wurden in 15-dimen\-sio\-nalen Rechnungen die Schwingungszust"ande von protonierten 
Wasserdimeren \cite{H5O2+MCTDH, H5O2+MCTDH2, H5O2+MCTDH3, H5O2+MCTDH4, H5O2+MCTDH5} untersucht. 
Schlie"s\-lich wurde in 12-dimensionalen Rechnungen die thermischen 
Geschwindig\-keitskonstanten \cite{HM1, HM2, WWM, SM, vHNM,NvHM}, anfangszustandsausgew"ahlte Reaktionswahrscheinlichkeiten
\cite{SM02, SM04, WeM5, WeM6, WeM8} und die 
state-to-state Reaktionswahrscheinlichkeiten \cite{WeM7} f"ur die Reaktion von Methan mit Wasserstoff 
untersucht. In diesen Rechnungen wurden detaillierte \textit{ab initio} berechneten
Potentialfl"achen verwendet. Signifikant h"ohere Dimensionen wurden in MCTDH-Rechungen mit 
Modelhamiltonoperatoren in Betracht gezogen. So wurde in wegweisende Rechnungen, in denen  
die nichtadibatischen Dynamiken von Pyrazin erforscht wurden, ein 24-moden 
schwingungsgekoppelter Hamiltonoperator \cite{WMC, WMC2, RWMC} verwendet. Multilayer-MCTDH Simulationen von typischen
physikalischen Modellen \cite{WT3, W1, WST, KCBWT, CTW2, WPHT} zu kondensierter Materie schlie"sen "ublich tausende Freiheitsgrade ein.
F"ur die Untersuchung eines Model von Photodissoziation in einem Wirt-Gast-Komplex wurden 189-
dimensionale ml-MCTDH Rechnungen \cite{WBRSM} durchgef"uhrt. F"ur Rezessionen, die das MCTDH-Verfahren und
seine Anwendungen diskutieren, siehe Refs. \cite{MCTDHreview, MCTDHreview2, HMreview1, MCTDHbook,Mreview2011, MCTDHreview3}.
\\ Im Rahmen dieser Masterarbeit wurde ein Python-Schnittstelle f"ur MCTDH entwickelt.
Python wurde als Anf"angerprogrammiersprache konzipiert, hat sich f"ur das schnelle Entwickeln von Anwendungssoftware bew"ahrt und wird zum 
Erstellen von Skript-Program\-men verwendet. \cite{PyBook} 
\\\textit{Low-level} Sprachen sind streng typisiert, um komplexe Aufgaben bew"ahltigen zu k"onnen. 
Auf der anderen Seite sind Skriptsprachen typenlos, wodurch die Verbindung von verschiedenen \textit{low-level} Komponenten erleichtern werden soll. \cite{PyBook2} 
Skript-Programme werden nicht kompiliert, sondern von einem Interpreter interpretiert. 
Der Speicher muss nicht vom Programmierer verwaltet werden. F"ur die Speicherverwaltung ist ein
\textit{Garbage collegtor} zust"andig.
 Mit Skript-Programme kann auf Module zugegriffen werden, die in \textit{low-level} Sprachen,
 wie C geschrieben sind. \cite{PyKana}     
 
 Der MCTDH-Quellcode besteht aus mehreren Komponenten, die in den Programmiersprachen Fortran77 und C++ geschrieben sind.
 Da Python sich insbesondere f"ur Programmier\-einsteiger eignet, wurde Python als Skriptsprache gew"ahlt, um Teile des MCTDH Quellcodes in Python aufrufen
zu k"onnen.

Aufgrund der gro"sen Menge an existierenden und gepr"uften Quellcode in Fortran und C, w"are das Umschreiben des Codes in Python
ein Verlust an wertvollen Ressourcen. In der Wissenschaft hat Python u. a. an Bedeutung gewonnen, da man 
mit Python in der Lage ist, existierende Komponenten in Python einzubetten, als diese neu schreiben zu m"ussen.
So besteht die Python-Bibliothek SciPy aus mehr als 200.00 Zeilen C++ Code, 60.000 Zeilen C Code und 75.000 Zeilen Fortran Code.
Dagegen umfasst der Python Code nur 70.000 Zeilen. \cite{PyArt} 
F"ur das Importieren von C, C++ und Fortran Code in Python eignet sich die Programmiersprache Cython, da sie eine Flachelernkurve
f"ur Python und C, C++ oder Fortran-Programmierer.
Anstelle eines Python-Interpreters wird der Cython-Code compiliert und in C-Code "ubersetzt.
Objekttypen aus C++ werden vom Cython-Compiler erkannt und der compilierte Cython-Code kann
einfach in Python importiert werden. \cite{PyArt}  

 Um den Rahmen dieser Arbeit nicht zu sprengen, wurde sich auf das Importieren von Objekttypen beschr"ankt, 
die die MCTDH Basis einlie"st. Die Basis des MCTDHs wird durch eine Baumstruktur angegeben, die nun in Python eingelesen werden kann.
\\Die MCTDH-Objekttypen, die in Python importiert wurden, wurden f"ur ein graphisches Benutzeroberfl"ache verwendet, das ebenfalls in
Python geschrieben wurde und die Eingabe der MCTDH-Basis erleichtern soll. Des Weiteren wurde eine "ubersichtliche Projektverwaltung 
von MCTDH-Rechnungen geschaffen.  
\\Diese Arbeit ist wie folgt gegliedert. In Kapitel 2 wird der Ansatz der MCTDH-Wellenfunk\-tion beschrieben. Es werden die Unterschiede zum ml-MCTDH herausgestellt.
Kapitel 3 stellt den Methodenteil dieser Arbeit dar und ist untergliedert in die Beschreibung der Python-Schnittstelle und der graphischen Benutzeroberfl"ache.     
Schlie"slich wird in Kapitel 4 eine Fazit gezogen und ein Ausblick gegeben.     