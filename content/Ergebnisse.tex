\chapter{Implementation und Demonstration}
\label{cha:Implementation}

\section{Sprachen und Bibliotheken}

Die Python-Schnittstelle f"ur MCTDH wurde in der Programmiersprache Cython \cite{PyArt} erstellt. 
Cython stellt eine Hybridprogrammiersprache dar,
die Python und C/C++ kombiniert. Im Gegensatz zu Python
wird Cython kompiliert, wobei die Cythonsyntax der Pythonsyntax "ahnlich ist.
In Cython kann C++ Code verwendet werden, der beim Erstellen der Cython-Programme ebenfalls kompiliert wird.

Die GUI f"ur die MCTDH-Rechnungen wurde in Python mithilfe der Qt-Bibliothek \cite{Qt} implementiert.
Der Zugriff auf die Qt-Bibliothek  erfolgt "uber die Python-Bibliothek PyQt4 \cite{PyQt}. 
Das Design der einzelnen GUI-Fenster erfolgte in Qt-Designer \cite{Qt-Designer}. 
Die in Qt-Designer erstellten Fenster werden in Python miteinander verkn"upft.

Neben PyQt wurden die Python-Module \textit{networkx} \cite{SciPyProceedings_11} und \textit{matplotlib} \cite{Hunter:2007} verwendet.
Mithilfe von Klassen aus \textit{networkx} und \textit{matplotlib} konnten Baumdiagramme erstellt und gespei\-chert werden, die in der GUI zur 
Visualisierung des MCTDH-Baums verwendet wurden.

\section{Python-Interface f"ur MCTDH}
\label{sec:PyInterface}

Es wurde eine Programmierschnittstelle (von englisch \textit{application programming interface}, API) f"ur das MCTDH-Programmpaket erstellt,
welche es erlaubt Funktionen des C++ MCTDH-Codes in Python-Skripten zu verwenden.
%Die Schnittstelle umfasst Klassen und Methoden, die zur Verwaltung der Darstellung der MCTDH-Wellenfunktion dienen.
Dazu wurden neue Klassen in Python erstellt, welche stellvertretende Klassen des C++ Codes sind 
und die einen eingeschr"ankten Satz an Funktionen bereitstellen.
Im Rahmen dieser Arbeit wurden Klassen erstellt, die der Verwaltung der Darstellung der MCTDH-Wellenfunktion die\-nen.
Die Erweiterung auf andere Klassen kann nach einem einfachen Schema erfolgen.
Abbildung \ref{fig:uml_Cython} zeigt ein Klassendiagramm aller in Cython erstellter Klassen der API.
Jede Klasse der API wird in Abbildung \ref{fig:uml_Cython} durch ein Rechteck repr"asentiert.
Im oberen Teil des Rechtecks ist der Klassenname angegeben und im unteren Teil sind die Methoden der Klasse aufgelistet.
Die baumartige Anordnung der Rechtecke repr"asentiert die Abh"angigkeit der Klassen zueinander. Eine Klasse h"angt von
allen Klassen ab, die sich im Diagramm unterhalb der betrachteten Klasse befinden und zu denen eine Verbindung besteht.
In der Klasse \textit{controlParameters} werden die Genauigkeitspara\-meter der MCTDH-Rechnung verwaltet. 
Die Klasse \textit{physCoor} verwaltet Informationen einer Koordinate und enth"alt Informationen "uber das primitive Gitter oder
die primitive Basis. Sie ist ein Memberobjekt der Klasse \textit{mctdhNode}, welche einen Knoten im MCTDH-Baum 
repr"asentiert. In der Klasse \textit{mctdhNode} wird die lokale Konnektivit"at des Knotens, sowie dessen Satz von
Entwicklungskoeffizienten verwaltet. Die Dimensionen der Entwicklungskoeffizienten sind in Objekten der Klasse
\textit{Tdim} gespeichert. Die Klasse \textit{mctdhBasis} enth"alt alle Objekte, die dem Management der Darstellung der
MCTDH-Wellenfunktion dienen.

Zur Demonstration der API wird im Folgenden ein Python-Skript vorgestellt, mithilfe dessen Dimensionsinformationen
einer Multilayer-MCTDH-Wellenfunktion berechnet werden k"onnen. Zuerst wird das Skript vorgestellt und 
anschlie"send wird es abschnittsweise erkl"art.

\begin{verbatim}
    import mctdh

    config = mctdh.controlParameters()
    config.initialize('mctdh.config')
    basis = mctdh.MctdhBasis()
    basis.initialize('basis.txt', config)
    
    maxNodes = basis.NmctdhNodes()
    
    nodes_spf = {}
    for i in range(maxNodes):
        node = basis.MCTDHnode(i)
        tdim = node.t_dim()
        nodes_spf[i] = tdim.GetnTensor() 
    
    primitivB = {i: basis.MCTDHnode(i).t_dim().active(0) for i in \
                range(maxNodes) if \
                    basis.MCTDHnode(i).Bottomlayer() == True}
    NCoefBottomNode = {}
    for key in primitivB:
        NCoefBottomNode[key] = primitivB[key] * nodes_spf[key]
    NCoefBottom = sum([l_[1] for l_ in NCoefBottomNode.items()])
    print NCoefBottom

    NCoefTopNode = 0
    for i in range(maxNodes):
        if basis.MCTDHnode(i).Toplayer() == True:
                children = basis.MCTDHnode(i).NChildren()
                for j in range(children):
                    NCoefTopNode *= basis.MCTDHnode(i).down(j).t_dim().GetnTensor()
                NCoefTopNode *= basis.MCTDHnode(i).t_dim().GetnTensor()
    print NCoefTopNode
    
    remnantNodeList = []
    remnant = 0
    for i in range(maxNodes):
        if basis.MCTDHnode(i).Toplayer() == False and \
        basis.MCTDHnode(i).Bottomlayer() == False:
                children = basis.MCTDHnode(i).NChildren()
                parent = basis.MCTDHnode(i).t_dim().GetnTensor()
                for j in range(children):
                    remnant *= basis.MCTDHnode(i).down(j).t_dim().GetnTensor() 
                remnant *= parent
                remnantNodeList.append(remnant)
                remnant = 0
    NCoefRemnant = sum(remnantNodeList)
    print NCoefRemnant
    
\end{verbatim}

Die in Abbildung \ref{fig:uml_Cython} dargestellten Klassen werden mit folgenden Befehl in Python importiert:

\begin{verbatim}
import mctdh
\end{verbatim}
Objekte der Klassen \textit{controlParameters} und \textit{mctdhBasis} werden instanziiert durch:
\begin{verbatim}
config = mctdh.controlParameters()
basis = mctdh.MctdhBasis()
\end{verbatim}
Die erstellten Objekte werden mittels
\begin{verbatim}
config.initialize('mctdh.config')
basis.initialize('basis.txt', config)
\end{verbatim}
initialisiert, wobei \textit{mctdh.config} und \textit{basis.txt} Dateinamen von Dateien sind, 
welche die Genauigkeitsparameter, bzw. die Definition der Darstellung der MCTDH-Wellenfunktion enthalten.

%Mithilfe des Objektes \textit{basis} kann die Anzahl der Knoten des eingelesenen MCTDH-Baums bestimmt werden:
Die Anzahl der Knoten im MCTDH-Baum wird mit dem folgenden Befehl bestimmt:
\begin{verbatim}
maxNodes = basis.NmctdhNodes()
\end{verbatim}
Danach wird der Maximalwert der $m$-Indizes der Entwicklungskoeffizienten $A^{l;\kappa_{1},...,\kappa_{l-1}}_{m;j_1,...,j_{d_{\kappa}}}(t)$ 
bestimmt (vergleiche Gleichung \ref{Eq:ml_mctdh_mode_SPF}).
Die entsprechenden Werte sind Member der Objekte der Klasse \textit{Tdim}.
Sie werden ermittelt und in dem Dictionary \textit{nodes\_spf} gespeichert:
\begin{verbatim}
for i in range(maxNodes):
    node = basis.MCTDHnode(i)
    tdim = node.t_dim()
    nodes_spf[i] = tdim.GetnTensor() 
\end{verbatim}

Eine Liste der Gr"o"sen aller primitiven Basiss"atze wird durch die folgenden Befehle erzeugt:
\begin{verbatim}
primitivB = {i: basis.MCTDHnode(i).t_dim().active(0) for i in \
            range(maxNodes) if \
            basis.MCTDHnode(i).Bottomlayer() == True}
\end{verbatim}

Daraufhin wird die Anzahl aller Entwicklungskoeffizienten der untersten Lage berechnet durch
\begin{verbatim}
for key in primitivB:
    NCoefBottomNode[key] = primitivB[key] * nodes_spf[key]
NCoefBottom = sum([l_[1] for l_ in NCoefBottomNode.items()]) .
\end{verbatim}

Die Anzahl der Entwicklungskoeffizienten der obersten Lage wird durch den folgenden Code-Abschnitt berechnet:

\begin{verbatim}
for i in range(maxNodes):
    if basis.MCTDHnode(i).Toplayer() == True:
            children = basis.MCTDHnode(i).NChildren()
            for j in range(children):
                NCoefTopNode *= basis.MCTDHnode(i).down(j).t_dim().GetnTensor()
            NCoefTopNode *= basis.MCTDHnode(i).t_dim().GetnTensor()
\end{verbatim}

Im letzten Code-Block wird die Summe der Anzahl der Entwicklungskoeffizienten aus
den "ubrigen Lagen gebildet:

\begin{verbatim}
for i in range(maxNodes):
        if basis.MCTDHnode(i).Toplayer() == False and \
        basis.MCTDHnode(i).Bottomlayer() == False:
                children = basis.MCTDHnode(i).NChildren()
                parent = basis.MCTDHnode(i).t_dim().GetnTensor()
                for j in range(children):
                    remnant *= basis.MCTDHnode(i).down(j).t_dim().GetnTensor() 
                remnant *= parent
                remnantNodeList.append(remnant)
                remnant = 0
    NCoefRemnant = sum(remnantNodeList)
\end{verbatim}

Schlie"slich wird die Anzahl aller Entwicklungskoeffizienten der unteren Knoten, des oberen Knoten
und der restlichen Knoten berechnet und ausgegeben:
\begin{verbatim}
 print NCoefBottom 
 print NCoefTopNode 
 print NCoefRemnant
\end{verbatim}
    
\begin{figure}
    \centering
    \includegraphics[scale=1]{figures/sequenceDiagram}
    \caption{Klassendiagramm der Programmierschnittstelle (API). Die Rechtecke repr"asentieren jeweils eine Klasse.
    Im oberen Bereich des Rechtecks befindet sich der Name der Klasse und im unteren Bereich eine Liste aller
    Memberfunktionen. F"ur eine detaillierte Beschreibung siehe Text.}\label{fig:uml_Cython}
\end{figure}



\section{Graphische Benutzeroberfl"ache f"ur MCTDH}

Es wurde eine GUI erstellt, von der aus MCTDH-Rechnungen gestartet werden k"onnen. 
Zur Demonstration der graphischen Benutzeroberfl"ache wird im folgenden das Vorgehen zum Starten einer 
MCTDH Simulation vorgef"uhrt. Als Beispiel dient die Realzeitpro\-pagation eines Wellenpakets auf dem
ersten elektronisch angeregten Zustand ($S_1$) von NOCl. 
Diese Rechnung ist zentraler Bestandteil bei der Berechnung der Photodisso\-ziation von NOCl \cite{MMC1}.
Es werden das gleiche Koordinatensystem, die gleiche Potentialfl"ache, sowie die gleichen numerischen Parameter wie in
[{\color{darkblue} 2}] benutzt.
Die Darstellung der MCTDH-Wellenfunktion ist in Abbildung \ref{fig:NOCl} dargestellt. 

\begin{figure}
    \centering
    \includegraphics[scale=0.5]{figures/NOCl}
    \caption{Darstellung der MCTDH-Wellenfunktion von NOCl.}\label{fig:NOCl}
\end{figure}

\begin{figure}
    \centering
    \includegraphics[scale=0.5]{figures/screenMain}
    \caption{Hauptfenster der MCTDH-GUI. Es zeigt aktuell vorhandene Projekte (\textbf{1}), sowie zugeh"orige Rechnungen (\textbf{2}).}\label{fig:workflow1}
\end{figure}

Beim Start der GUI erscheint das Hauptfenster (siehe Abbildung \ref{fig:workflow1}).
In Liste \textbf{1} sind alle vorhandenen Projekte aufgef"uhrt.
Nach Auswahl eines Projekts erscheinen in Liste \textbf{2} alle dem Projekt zugeordneten Rechnungen.
Durch die Schaltfl"achen ,,+'' und ,,-'' k"onnen Projekte sowie Rechnungen jeweils hinzugef"ugt bzw.
entfernt werde.
Im Reiter ,,File'' (\textbf{3}) befinden sich zus"atzliche Optionen zur Projektverwaltung (siehe Abbildung \ref{fig:workflow2}).
Durch Klicken der Schaltfl"ache \textbf{5} wird ein neues Projekt erstellt. Externe Projekte werden durch die 
Schaltfl"ache \textbf{6} geladen. Bet"atigung der Schaltfl"ache \textbf{7} beendet die Benutzeroberfl"ache.

\begin{figure}
    \centering
    \includegraphics[scale=0.5]{figures/screenMainFile}
    \caption{Hauptfenster der MCTDH-GUI. Nach Bet"atigung des Reiters ,,File''
		stehen Optionen zur Projektverwaltung, sowie zum Beenden der GUI zur Verf"ugung.}\label{fig:workflow2}
\end{figure}

Zum Starten einer neuen Rechnung wird zuerst der Reiter ,,Project'' (Abbildung \ref{fig:workflow1} \textbf{4}) 
ausgew"ahlt. Zwei neue Buttons erscheinen (Abbildung \ref{fig:workflow3}). Durch Klicken von \textbf{8} 
erscheint ein neues Fenster mit dem Namen ,,MCTDH calculation'' (siehe Abbildung \ref{fig:workflow4}). 
In diesem Fenster wird die neue MCTDH Rechnung spezifiziert.

\begin{figure}
    \centering
    \includegraphics[scale=0.5]{figures/screenMainProject}
    \caption{Hauptfenster der MCTDH-GUI nach Bet"atigung des Reiters ,,Project''. Es stehen Optionen zum Erstellen und Starten einer
    MCTDH-Rechnung zur Verf"ugung.}\label{fig:workflow3}
\end{figure}

In dem Eingabefenster ,,MCTDH calculation'' kann im Feld \textbf{10} der Name der Rechnung angegeben werden. 
Sollte beim Speichern (Abbildung \ref{fig:workflow4} \textbf{18}) des MCTDH-Baums und der Einstellungsparameter 
kein Name angegeben sein, wird der Benutzer 
aufgefordert einen Name f"ur die Rechnung zu w"ahlen. 
%Bevor die Rechnung gespeichert wird, 
%wird "uberpr"uft, ob der gew"ahlte Name dem Namen einer andere Rechnung entspricht und, ob diese 
%dann gegebenfalls "uberschrieben werden soll.
Beim Speichern der Rechnung wird "uberpr"uft, ob eine Rechnung mit dem ausgew"ahlten Namen existiert,
um ein unabsichtliches "Uberschreiben zu verhindern.
In Liste \textbf{11} wird der entsprech\-ende Summe-von-Produkten Operator f"ur das gegebene System ausgew"ahlt.
Dabei werden die primitiven Basisparameter festgelegt.
"Uber die Kn"opfe  ,,on'' und ,,off'' (\textbf{12}) wird gesteuert, ob eine Potentialfl"achenauswertung mittels 
CDVR erfolgt.
Ist der ,,off''-Knopf aktiviert, werden keine Potentiale in der Liste \textbf{13} angezeigt.
Bei der Standardeinstellung der GUI ist der Knopf ,,on'' aktiviert und es k"onnen Potentiale durch Klicken
auf die Elemente der Liste \textbf{13} ausgew"ahlt werden. Im vorliegenden Beispiel wird die Option ,,off''
ausgew"ahlt.
Die Erweiterung der Summe-von-Produkten Operatoren und Potentialfl"achen kann schematisch erfolgen.

%Mithilfe der MCTDH-GUI k"onnen verschiedene Operationen durchgef"uhrt werden (\textbf{14}).
In Liste \textbf{14} wird die Art der Rechnung ausgew"ahlt. Im vorliegenden Beispiel wird ,,real-time propagation'' 
zum Durchf"uhren einer Realzeitpropagation ausgew"ahlt.
Weite\-re M"oglichkeiten sind ,,imaginary-time propagation'' zum Durchf"uhren einer Imagin"arzeitpropagation 
sowie ,,Eigenstate calculation'' zum Berechnen von Eigenzust"anden und 
,,Thermal flux eigenstate calculation'' zur Berechnung von thermischen Flusseigenzust"an\-den \cite{thermalflux}.
Des Weiteren k"onnen unter \textbf{15} Steuerungsparameter f"ur den Integrator angegeben werden. Dazu geh"oren die Anfangszeit, die Endzeit,
der initiale Zeitschritt und die Anzahl an Iterationen. 
Alle Parameter werden automatisch geladen, indem eine Eingabe-Datei durch 
die ,,Load''-Schaltfl"alche \textbf{19} eingelesen wird. 
Der MCTDH-Baum, welcher die aktuell gew"ahlte Basis repr"asentiert, wird unter \textbf{16} 
als Listendiagramm angezeigt. Hier
kann durch Doppelklick die Anzahl der SPFs, der primitiven Basis und der Mo\-den ver"andert werden.
Zus"atzlich wird der MCTDH-Baum in Form eines Diagramms in \textbf{17} angezeigt. 
Die Schaltfl"ache ,,Cancel'' \textbf{21} beendet das Eingabefenster
und die Schaltfl"ache ,,Start calculation'' (\textbf{20}) beginnt die MCTDH-Simulation.
Alternativ kann die MCTDH-Simulation auch aus dem Hauptfenster gestartet werden,
indem Schaltfl"ache ,,Run MCTDH on existing job'' (\textbf{9}) bet"atigt wird (siehe Abbildung \ref{fig:workflow3}).
%Durch Klicken der Schaltfl"ache ,,Run MCTDH on existing job'' (\textbf{9}) kann ebenfalls eine MCTDH Simulation gestartet.

Abbildung \ref{fig:workflow5} zeigt das Eingabefenster, in dem nach Auswahl aller Parameter die oben genannte Rechnung 
durch Klicken der Schaltfl"ache \textbf{20} gestartet wird.

\begin{figure}
    \centering
    \includegraphics[angle=90, scale=0.45]{figures/screenWidgetA}
    \caption{Spezifikation einer neuen MCTDH-Rechnung durch die MCTDH-GUI.}\label{fig:workflow4}
\end{figure}
\begin{figure}
    \centering
    \includegraphics[angle=90, scale=0.45]{figures/screenWidgetAexample}
    \caption{Eingabefenster der MCTDH-GUI am Beispiel der
    Photodissoziation von NOCl.}\label{fig:workflow5}
\end{figure}
