\chapter{Fazit und Ausblick}

in dieser Arbeit wurde ein MCTDH-Python-Modul erstellt, das Klassen des MCTDH-Codes in Python nutzbar macht.
In Zukunft k"onnten weitere Teile des MCTDH-Codes in Python zug"anglich gemacht werden, sodass weitere Funktionen
des MCTDH-Codes in Python verwendet werden ko"nnen.


Die MCTDH-GUI kann erweitert werden. Eine Auswahl an Basisdateien f"ur verschiedene Hamiltonoperatoren existiert bereits. Der n"achste Schritt
w"are die M"oglichkeit, dass der Benutzer die Hamiltonoperatoren durch das Anklicken vorgegebener Terme selber erstellt. 
Des Weiteren k"onnte ein Ausgabefenster erstellt werden, das die Zwischenergebnisse w"ahrend der MCTDH-Rechnung graphisch darstellt.
Die Drag\&Drop-Funktion k"onnte f"ur das Baumdiagramm in Abbildung \ref{fig:workflow4} hinzugef"ugt werden, sodass die MCTDH-B"aum
graphisch erstellt werden k"onnten.
Schlie"slich w"are auch das Implementieren der MCTDH-GUI als App auf Smartphones f"ur die Benutzer hilf\-reich, von dem sie die laufenden Rechnungen
"uberpr"ufen k"onnten. 