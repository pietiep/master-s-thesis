\chapter{Zusammenfassung und Ausblick}
\label{cha:fazit}

In dieser Arbeit wurde eine Pythonschnittstelle f"ur den in C++ verfassten MCTDH-Code erstellt, 
welche es erlaubt Klassen in Python-Skripten zu nutzen, die der Verwaltung der MCTDH-Basis dienen. 
In Zukunft kann die Schnittstelle um weitere Klassen des MCTDH-Codes erg"anzt werden, sodass sie
vielseitig eingesetzt werden kann.

Eine graphische Benutzeroberfl"ache wurde erstellt, welche es Nutzern ohne Programmierkenntnisse
erlaubt MCTDH-Rechnungen zu starten. Im Rahmen dieser Arbeit wurde in der GUI eine Auswahl von
Summe-von-Produkten Operatoren, sowie von \textit{ab initio} Potentialfl"achen bereitgestellt.
Diese Auswahl kann in Zukunft erweitert werden. Zudem w"are eine Funktion zum Erstellen von
benutzerdefinierten Operatoren mittels Drag\&Drop hilfreich. Die Struktur der MCTDH-Basis
wird aktuell "uber externe Dateien eingelesen; auch hier w"are es sinnvoll eine Funktion zum
Erstellen und Modifizieren der Multilayer-MCTDH Basisstruktur mittels Drag\&Drop bereitzustellen.

Die Funktionen der GUI konzentriert sich bisher auf das Erstellen und Verwalten von
projektbezogenen Rechnungen. Es w"are n"utzlich die Funktionen auf das Verfolgen und Visualisieren
laufender oder beendeter Rechnungen auszuweiten. Dazu ist es sinnvoll f"ur die Ausgabe des C++ 
MCTDH-Programms ein standardisiertes Protokollformat zu entwickeln, welches von externen Programmen eingelesen
und interpretiert werden kann. Somit w"are beispielsweise ein Einbinden von Smartphones und Tabletts
denkbar, um laufende Rechnungen unkompliziert und leicht verfolgen zu k"onnen.

