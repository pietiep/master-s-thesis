\chapter{Zusammenfassung und Ausblick}
\label{cha:fazit}

In dieser Arbeit wurden zentrale Schritte ergriffen, um die MCTDH-Methode, welche bisher gr"o"stenteils
Spezialisten vorenthalten war, f"ur einen gr"o"seren Anwenderkreis nutz\-bar zu machen.
Zuerst wurde aus den vielseitigen Anwendungsgebieten der MCTDH-Methode ein Satz generischer
Operationen ermittelt, deren Nutzbarmachung f"ur eine breitere Anwendergruppe sinnvoll ist.
Dann wurden zwei wesentliche Schritte ergriffen, um diese Operationen auch
Wissenschaftlern zug"anglich zu machen, die keine tiefere Kenntnis der Methode und
"uber keine, oder nur eingeschr"ankte Programmierkenntnisse verf"ugen.

Zum einen wurde eine graphische Benutzeroberfl"ache erstellt, welche es Nutzern ohne
Programmierkenntnisse erlaubt MCTDH-Rechnungen zu starten. Dazu steht eine Aus\-wahl
von Summe-von-Produkten Operatoren, sowie von \textit{ab initio} Potentialfl"achen bereit.
Mithilfe der GUI k"onnen Wellenpaketspropagationen, sowie die Berechnung von Eigenzust"anden
und thermischen Flusseigenzust"anden durchgef"uhrt werden, was vielseitige Einsatzm"oglichkeiten
zul"asst. Dazu z"ahlen unter anderem die Berechnung von Photodissoziationsspektren, 
Schwingungsenergien, thermischen Raten und vielem mehr.

Zum anderen wurde eine Pythonschnittstelle f"ur den in C++ verfassten MCTDH-Code erstellt, 
welche es erlaubt Klassen in Python-Skripten zu nutzen, die der Verwaltung der MCTDH-Wellenfunktion dienen. 
Dadurch k"onnen Informationen zur Darstellung der MCTDH-Wellenfunktionen (z.B. Skalierungseigenschaften
einer geplanten Rechnungen) mithilfe von Python-Skripten berechnet werden.

Die in dieser Arbeit erstellte GUI bietet viele sinnvolle Ankn"upfungspunkte zur Erweiterung.
Die Auswahl der Summe-von-Produkten-Operatoren, sowie der \textit{ab initio} Potentialfl"achen kann in Zukunft
erweitert werden. Zudem w"are eine Funktion zum Erstellen von benutzerdefinierten Operatoren
hilfreich, um die Simulation neuer Systeme zuzulassen. Momentan wird die Baumstruktur der MCTDH-Wellenfunk\-tion
"uber externe Dateien eingelesen; auch hier w"are es sinnvoll eine Funktion zum
Erstellen und Modifizieren der Multilayer-MCTDH Basisstruktur (zum Beispiel mit\-tels Drag\&Drop) bereitzustellen.

Die Funktionen der GUI konzentriert sich bisher auf das Erstellen und Verwalten von
projektbezogenen Rechnungen. Es w"are n"utzlich die Funktionen auf das Verfolgen und Visualisieren
laufender oder beendeter Rechnungen auszuweiten. Dazu ist es sinnvoll f"ur die Ausgabe des C++ 
MCTDH-Programms ein standardisiertes Protokollformat zu entwickeln, welches von externen Programmen eingelesen
und interpretiert werden kann. Somit w"are beispielsweise ein Einbinden von Smartphones und Tabletts
denkbar, um laufende Rechnungen unkompliziert und leicht verfolgen zu k"onnen.

In Zukunft kann die Python-Schnittstelle um weitere Klassen des MCTDH-Codes erg"anzt werden, sodass diese
vielseitiger eingesetzt werden k"onnen. Ein sinnvoller erster Schritt w"are die Erweiterung auf Klassen,
welche der Verwaltung der MCTDH-Wellenfunktion dienen. Dadurch w"aren beispielsweise Modifikation der
Darstellung der Wellenfunktion m"oglich.
