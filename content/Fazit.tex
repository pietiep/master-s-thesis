\chapter{Fazit und Ausblick}

in dieser Arbeit wurde ein MCTDH-Python-Modul erstellt, das Klassen des MCTDH-Codes in Python nutzbar macht.
In Zukunft k"onnten weitere Teile des MCTDH-Codes in Python zug"anglich gemacht werden, sodass weitere Features 
des MCTDH-Codes in Python verwendet werden kann.
Es w"are denkbar, dass n"utzliche Python-Module wie NumPy (numerical Python), SciPy (Scientific Computing Tools for Python)\cite{SciPy}
und TensorFlow (machine learning) \cite{TensorFlow} nahtlos mit dem MCTDH-Modul genutzt werden k"onnte. 

Auch die MCTDH-GUI kann erweitert werden. Eine Auswahl an Basisdateien f"ur verschiedene Hamiltonoperatoren existiert bereits. Der n"achste Schritt
w"are die M"oglichkeit, dass der Benutzer die Hamiltonoperatoren durch das Anklicken vorgegebener Terme selber erstellt. 
Des Weiteren k"onnte ein Ausgabefenster erstellt werden, das die Zwischenergebnisse w"ahrend der MCTDH-Rechnung graphisch darstellt.
Schlie"slich w"are auch das Implementieren der MCTDH-GUI als App auf Smartphones f"ur die Benutzer hilf\-reich, von dem sie die laufenden Rechnungen
"uberpr"ufen k"onnten. 