\chapter{Anhang}

dialogC.py:
\begin{verbatim}
    from PyQt4 import QtCore, QtGui, uic
    from PyQt4 import *
    import sys, os
    from widgetA import WidgetA
    
    base, form = uic.loadUiType("dialogNewName.ui")
    
    class DialogC(base, form):
        def __init__(self, parent=None):
            super(base, self).__init__(parent)
            self.setupUi(self)
    
            self._FolderName = ""
            self._cancel = None
    
            self.uiWarning.setText('')
            self.uiFolderName.setText("Give name for new Folder!")
            self.uiFolderName.selectAll()
            self.uiFolderName.textChanged.connect(self.change0)
            self.uiSaveBu.clicked.connect(self.save)
            self.uiCancelBu.clicked.connect(self.esc)
        
        def setWarning(self, warningUi):
            print warningUi
            self.uiWarning.setText(warningUi)
    
        def save(self):
            self._FolderName = self.uiFolderName.text()
            if ' ' not in self._FolderName:
                self.close()
            else:
                self.setWarning(str('Name contains whitespace'))
                print 'Name contains whitespace'
    
        def esc(self):
            self._cancel = "Cancel"
            self.close()
    
        def change0(self):
            self._FolderName = self.uiFolderName.text()
\end{verbatim}

InputPro.py:

\begin{verbatim}
from PyQt4 import QtCore, QtGui
import sys, os

class ListAbstrModel(QtCore.QAbstractListModel): 
    def __init__(self, data=[], parent=None):
        super(ListAbstrModel, self).__init__(parent)
        self.__data = data

    def rowCount(self, parent):
        return len(self.__data)

    def data(self, index, role):
        if role == QtCore.Qt.EditRole:
            row = index.row()
            return self.__data[row]

        if role == QtCore.Qt.DisplayRole:
            row = index.row()
            value = self.__data[row]
            return value

    def flags(self, index):
        return QtCore.Qt.ItemIsEditable |QtCore.Qt.ItemIsEnabled |QtCore.Qt.ItemIsSelectable

    def setData(self, index, value, role=QtCore.Qt.EditRole):
        if role == QtCore.Qt.EditRole:
            row = index.row()
            self.__data[row] = str(value.toString())
            return True
        return False

class ListModel(ListAbstrModel):
    def __init__(self, data=[], parent=None):
        super(ListModel, self).__init__(data, parent)
        self.__data = data
        self.__dataBefore = list(data)
        self._messageBu = False

    def setData(self, index, value, role=QtCore.Qt.EditRole):
        if value not in self.__data:
            if role == QtCore.Qt.EditRole:
                row = index.row()
                self.__data[row] = str(value.toString())
                self.getValue()
                return True
        else:
            return False

    def getValue(self):
        matches = list(set(self.__data).intersection(self.__dataBefore))
        new = [l_ for l_ in self.__data if l_ not in matches]
        old = [l_ for l_ in self.__dataBefore if l_ not in matches]
        print new, old
        print os.getcwd()
        self.__dataBefore = list(self.__data)
        try:
            os.rename(old[0], new[0])
        except OSError:
            raise

    def removeRows(self, position, rows, parent=QtCore.QModelIndex()):
        self.beginRemoveRows(parent, position, position+rows-1)

        value = self.__data[position]
        self.showdialog(value)
        if self._messageBu == 'OK':
            for i in range(rows):    
                value = self.__data[position]   
                self.__data.remove(value)
            self.endRemoveRows()
            return True
        else:
            self.endRemoveRows()
            return False
            
    def showdialog(self, value):
        msg = QtGui.QMessageBox()
        msg.setIcon(QtGui.QMessageBox.Warning)

        msg.setText("Are sure you want to delete Folder %s?" %value)
        msg.setStandardButtons(QtGui.QMessageBox.Ok| QtGui.QMessageBox.Cancel)

        msg.buttonClicked.connect(self.msgbtn)
        msg.exec_()

    def msgbtn(self, i):
        self._messageBu = str(i.text())
        
class ListModel2(ListModel):  
    def __init__(self, project, data=[], parent=None):
        super(ListModel2, self).__init__(data, parent)
        self.__data = data
        self.__dataBefore = list(data)
        self._changePath = os.getcwd() + '/' + project

    def setData(self, index, value, role=QtCore.Qt.EditRole):
        if value not in self.__data:
            if role == QtCore.Qt.EditRole:
                row = index.row()
                self.__data[row] = str(value.toString())
                self.getValue(row)
                return True
        else:
            return False

    def getValue(self, row):
        matches = list(set(self.__data).intersection(self.__dataBefore))
        new = [l_ for l_ in self.__data if l_ not in matches]
        old = [l_ for l_ in self.__dataBefore if l_ not in matches]
        self.__dataBefore = list(self.__data)
        os.chdir(self._changePath)
        try:
            os.rename(old[0], new[0])
        except Exception:
            self.showdialog(row)

    def removeRows(self, position, rows, parent=QtCore.QModelIndex()):
        self.beginRemoveRows(parent, position, position+rows-1)

        value = self.__data[position]
        for i in range(rows):    
            value = self.__data[position]   
            self.__data.remove(value)
        self.endRemoveRows()
        return True
\end{verbatim}

InputTree.py:

\begin{verbatim}
##############################################################    
#    Title: PyQt4 Model View Tutorial Part 04
#    Author: Yasin Uludag
#    Date: 2011
#    Availability: https://www.youtube.com/watch?v=pr1M3mP7qfI
##############################################################

from PyQt4 import QtCore, QtGui, uic
import sys
from Node import Node, BottomNode, Tree

class SceneGraphModel(QtCore.QAbstractItemModel):
    def __init__(self, root, parent=None):
        super(SceneGraphModel, self).__init__(parent)
        self._rootNode = root
        self._child = None
        self._childIndex = QtCore.QModelIndex()
        self._dictNodes = {}

    def parent(self, index):
        node = index.internalPointer()
        parentNode = node.parent()

        if parentNode == self._rootNode:
            return QtCore.QModelIndex()
        return self.createIndex(parentNode.row(), 0, parentNode)

    def index(self, row, column, parent):


        childItem = parentNode.child(row)

        if childItem:
            return self.createIndex(row, column, childItem)
        else:
            return QtCore.QModelIndex()

    def rowCount(self, parent):
        if not parent.isValid():
            parentNode = self._rootNode
        else:
            parentNode = parent.internalPointer()

        return parentNode.childcount()

    def columnCount(self, parent):
        return 2

    
    def flags(self, index):
        return QtCore.Qt.ItemIsEnabled | QtCore.Qt.ItemIsSelectable | \
            QtCore.Qt.ItemIsEditable | QtCore.Qt.ItemIsDragEnabled | \
            QtCore.Qt.ItemIsDropEnabled

    def headerData(self, section, orientation, role):
        if role == QtCore.Qt.DisplayRole:
            if section == 0:
                return "Node"
            else:
                return "Mode"

    def data(self, index, role):

        if not index.isValid():
            return None

        node = index.internalPointer()

        if role == QtCore.Qt.DisplayRole or role == QtCore.Qt.EditRole:
            if index.column() == 0:
                return node.name()
            else:
                if node.typeInfo() == "Bottom":
                    return node._physcoor

        if role == QtCore.Qt.DecorationRole:
            if index.column() == 0:

                typeInfo = node.typeInfo()

    def setData(self, index, value, role = QtCore.Qt.EditRole):
        if index.isValid():
            if role == QtCore.Qt.EditRole:
                node = index.internalPointer()
                if index.column() == 0:
                    node.setName(str(value.toString()))
                    return True
                elif index.column() == 1:
                    node.setPhyscoor(str(value.toString()))
                    return True
        return False

    def getNode(self, index):
        if index.isValid():
            node = index.internalPointer()
            if node:
                return node
        return self._rootNode

    def getNode2(self, index):
        return self._rootNode

    def insertRows(self, position, rows, parent=QtCore.QModelIndex()):
        parentNode = self.getNode(parent)
        oldNode = self._child
        if oldNode.typeInfo() == 'NODE':
            self.addNode(oldNode.name(), oldNode.name(), None)    
        if oldNode.typeInfo() == 'Bottom':
            self.addBottomNode(oldNode.name(), oldNode.name(), None, oldNode.physcoor())
        self.copyNode(self._child)

        self.beginInsertRows(parent, position, position+rows-1)

        for row in range(rows):
            success = parentNode.insertChild(position, self._dictNodes[oldNode.name()])  

        self.endInsertRows()

        return success

    def removeRows(self, position, rows, parent=QtCore.QModelIndex()):
        parentNode = self.getNode(parent)
                
        self.beginRemoveRows(parent, position, position+rows-1)

        for row in range(rows):
            
            success = parentNode.removeChild(position)

        self.endRemoveRows()

        return success

#####Drag and Drop######

    def supportedDropActions(self):
        return QtCore.Qt.CopyAction | QtCore.Qt.MoveAction

    def mimeTypes(self):
        types = QtCore.QStringList()
        types.append('text/plain')
        return types

    def mimeData(self, index):  
        rc = ''
        self._child = index[0].internalPointer()
        self._childIndex = index[0]
        mimeData = QtCore.QMimeData()
        mimeData.setText(rc)
        return mimeData

    def dropMimeData(self, data, action, row, column, parentIndex):
        if action == QtCore.Qt.IgnoreAction:
            return True
        
        self.insertRows(0,1, parentIndex)
        index = self._childIndex
        self.removeRows(0,1, index)
        print self._rootNode.child(0)
        return True

    def copyNode(self, oldNode):
        children = oldNode.childAll()
        if children:
            for oldchild in children:
                if oldchild.typeInfo() == "NODE":
                    self.addNode(oldchild.name(), oldchild.name(), self._dictNodes[oldNode.name()])
                if oldchild.typeInfo() == "Bottom":
                    self.addBottomNode(oldchild.name(), oldchild.name(), self._dictNodes[oldNode.name()], oldchild.physcoor())
                self.copyNode(oldchild)
    
    def addNode(self, obj, SPF, parent):
        self._dictNodes[obj] = Node(SPF, parent)

    def addBottomNode(self, obj, SPF, parent, physcoor):
        self._dictNodes[obj] = BottomNode(SPF, parent, physcoor)
\end{verbatim}

LogicalNodes.py:

\begin{verbatim}
from ModelTree import ModelTree
import networkx as nx
from networkx.drawing.nx_agraph import write_dot, graphviz_layout
import matplotlib.pyplot as plt

class LogicalNodes():
    def __init__(self, layer_matr, config_file, sys_file):
        self.G = nx.DiGraph()
        self.layer_matr = layer_matr
        self.ModelTree = ModelTree(config_file, sys_file)
        self.label_mode = self.ModelTree.label_mode
        self.nodes_spf = self.ModelTree.nodes_spf
        self.Networkx()
        self.augDiGraph("SPF", self.nodes_spf)
        self.augDiGraph("Mode", self.label_mode)

    def Networkx(self):
        for c_ in self.layer_matr:
            for b_ in c_:
                self.G.add_node(b_)
                l = len(c_)
                for index, b_ in enumerate(c_):
                    if index < (l-1):
                        self.G.add_edge(b_, c_[index + 1], weight=1)

    def augDiGraph(self, str_kind, str_dict):
        for key, ele_ in str_dict.items():
            self.G.nodes[key][str_kind] = ele
\end{verbatim}

main.py:

\begin{verbatim}
    from PyQt4 import QtGui, QtCore, uic
    import sys, os, shutil
    from widgetA import WidgetA
    from InputPro import ListModel, ListModel2
    from dialogC import DialogC
    from dialogD import DialogD
    
    
    base, form = uic.loadUiType("main.ui")
    
    class Main(base, form):
        def __init__(self, parent=None):
            super(base, self).__init__(parent)
            self.setupUi(self)
    
            self._HamiltonianDir = os.getcwd() + '/' + 'Hamiltonians'
            os.chdir('Projects')
            self._startingPath = os.getcwd()
            self._ProjectName = None
            self._path2 = None
            self._dir_list = None
            self._proContent = []
    
            self._model1 = None
            self._model2 = None
            self._itemIndex1 = None
            self._itemIndex2 = None
    
            self.getdirs()
            self._WidgetA = WidgetA(self)
            self._WidgetA._HamiltonianDir = self._HamiltonianDir
            self._WidgetA._startingPath = self._startingPath
    
            self.setList()
            self.uiNew.triggered.connect(self.openA)
            self.uiLoad.triggered.connect(self.openB)
            self.uiMCTDHcalc.triggered.connect(self.openD)
            self.uiMCTDHexisting.triggered.connect(self.runJob)
            self.uiPlusBu.clicked.connect(self.openA)
            self.uiPlusBu2.clicked.connect(self.open0)
            self.uiMinusBu.clicked.connect(self.removeA)
            self.uiMinusBu2.clicked.connect(self.remove0)
    
            self.uiProjects.clicked.connect(self.on_item_select)
            self.uiProjects.customContextMenuRequested.connect(self.openMenu)
    
            self.setList2()
            self.uiSessions.clicked.connect(self.on_item_select0)
            self.uiSessions.customContextMenuRequested.connect(self.openMenu0)
    
            self._dialogD = DialogD()
    
            self._messageBu = None
    
        def openMenu0(self, position):
            """Context menu"""
            menu = QtGui.QMenu()
            renameAction = menu.addAction("Rename")
            action = menu.exec_(self.uiSessions.mapToGlobal(position))
            if action == renameAction:
                self.uiSessions.edit(self._itemIndex2)
    
        def openMenu(self, position):
            """Context menu"""
            menu = QtGui.QMenu()
            renameAction = menu.addAction("Rename")
            action = menu.exec_(self.uiProjects.mapToGlobal(position))
            if action == renameAction:
                self.uiProjects.edit(self._itemIndex1)
    
        def remove0(self):
            """Removes Rows from ListModel2()"""
            rowNum = self._itemIndex2.row()
            session = str(self._itemIndex2.data().toString())
            self.showdialog(session)
            if 'OK' in self._messageBu:
                self._model2.removeRows(rowNum,1, self._itemIndex2)
    
                shutil.rmtree(self._startingPath+'/'+self._ProjectName+'/'+session)
    
        def removeA(self):
            """Removes Rows from ListModel()"""
            rowNum = self._itemIndex1.row()
            key = str(self._itemIndex1.data().toString())
            startingpath2 = self._startingPath + '/'
            delFolder = startingpath2 + key
    
            self._model1.removeRows(rowNum,1, self._itemIndex1)
    
            if self._model1._messageBu == 'OK' and delFolder != self._startingPath and delFolder != startingpath2:
                shutil.rmtree(self._startingPath+'/'+key)
    
                maxRow = self._model2.rowCount(self._itemIndex2)
                if maxRow != 0:
                    self._model2.removeRows(0, maxRow)
    
        def showdialog(self, value):
            msg = QtGui.QMessageBox()
            msg.setIcon(QtGui.QMessageBox.Warning)
    
            msg.setText("Are sure you want to delete Folder %s?" %value)
            msg.setStandardButtons(QtGui.QMessageBox.Ok| QtGui.QMessageBox.Cancel)
    
            msg.buttonClicked.connect(self.msgbtn)
            msg.exec_()
    
        def msgbtn(self, i):
            self._messageBu = str(i.text())
    
        def getContent(self):
            os.chdir(self._startingPath)
            if os.path.exists(self._ProjectName):
                directories = os.walk('./'+self._ProjectName).next()[1]
                try:
                    directories.remove('tmp') 
                except ValueError:
                    pass
                self._proContent = sorted(directories)
            else:
                print("path doesn't exists")
    
        def on_item_select0(self, index):
            """clicked Event on Items belonging to ListModel2()"""
            os.chdir(self._startingPath)
            projectFolder = str(self._itemIndex1.data().toString())
            sessionFolder = str(index.data().toString())
            self._itemIndex2 = index
            self._ProjectName = projectFolder
    
            os.chdir(self._ProjectName)
            self._WidgetA._ProjectName = projectFolder
            self._WidgetA.makedir()
            self._WidgetA.editSession(sessionFolder)
            self._WidgetA.setSessionName(sessionFolder)
            self._WidgetA.start()
            self._WidgetA.removeContent()
            SESfiles = os.walk(self._startingPath+'/'+self._ProjectName+'/'+sessionFolder).next()[2]
            if SESfiles:
                for f_ in SESfiles:
                    if 'txt' in f_:
                        self._WidgetA.fromSESToTMP(f_)
            self.openC()
            os.chdir('./')
    
        def on_item_select(self, index):
            """clicked Event on Items belonging to ListModel()"""
            self._itemIndex1 = index
            self._ProjectName = str(index.data().toString())
            self._WidgetA._ProjectName = self._ProjectName
            self._WidgetA._SessionName = None
            self._WidgetA.clearSession()
            self.getdirs()
            self.setList2()
    
        def setList2(self):
        #####ListModelPES#######
            self.getContent()
            self._model2 = ListModel2(self._ProjectName, self._proContent)
            self.uiSessions.setModel(self._model2)
            indices = self.uiSessions.selectionModel().selectedIndexes()
            if not indices:
                index = self._model2.index(0,0)
                self._itemIndex2 = index
                self.uiSessions.selectionModel().select(index, QtGui.QItemSelectionModel.Select)
                self._path2 = str(self._itemIndex2.data().toString())
    
        def setList(self):
        #####ListModelPES#######
            self.getdirs() 
            self._model1 = ListModel(self._dir_list)
            self.uiProjects.setModel(self._model1)
            indices = self.uiProjects.selectionModel().selectedIndexes()
            if not indices:
                index = self._model1.index(0,0)
                self._ProjectName = str(index.data().toString())
                self._WidgetA._ProjectName = self._ProjectName
                self._itemIndex1 = index
                self._itemProxyIndex1 = index
                self.uiProjects.selectionModel().select(index, QtGui.QItemSelectionModel.Select)
    
        def getdirs(self):
                directories = os.walk(self._startingPath).next()[1]
                self._dir_list = [dirs for dirs in directories]
                self._dir_list = sorted(self._dir_list)
    
        def open0(self, warn):
            if warn == False:
                warn = ''
            dialogC = DialogC()
            dialogC.setWarning(str(warn))
            dialogC.exec_()
            self._path2 = str(dialogC._FolderName)
            print self._path2
    
            if str(dialogC._cancel) != 'Cancel':
                path = self._startingPath + '/' + self._ProjectName + '/' + self._path2
                if not os.path.exists(path):
                    try:
                        os.chdir(self._startingPath+'/'+self._ProjectName)
                        os.makedirs(self._path2)
                        os.chdir(self._startingPath)
                    except IOError as identifier:
                        print (dir(identifier))
                    self.getContent()
                    self.setList2()
                else:
                    print('Folder already exists!')
                    self.open0('Folder already exists!')
    
        def openA(self, warnings):
            if warnings == False:
                warnings = ''
            dialogC = DialogC()
            dialogC.setWarning(str(warnings))
            dialogC.exec_()
            self._ProjectName = str(dialogC._FolderName)
    
            if self._ProjectName != 'Cancel':
                path = self._startingPath + '/' + self._ProjectName
   
                if not os.path.exists(path):
                    try:
                        os.makedirs(path)
                        os.makedirs(path+'/tmp')
                    except IOError as identifier:
                        print (identifier)
   
                    self.setList()
                else:
                    print('Folder already exists!')
                    self.openA('Folder already exists!')
    
        def openB(self):
            self._ProjectName = str(QtGui.QFileDialog.getExistingDirectory(self))
            self._ProjectName = self._ProjectName.split("/")[-1]
            self.getContent()
            self.setList2()
    
        def openC(self):
            dialog = self._WidgetA
            dialog.exec_()
            self.setList2()
    
        def openD(self):
            self._WidgetA.clearSession()
            self._WidgetA.setSessionName(None)
            self._WidgetA.removeContent()
    
            self.openC()
    
        def Finder(self, path, app):
            fileList = os.walk(path).next()[2]
            return [f_ for f_ in fileList if app in f_][0]
    
        def getRunInput(self):
            filePathList = []
            self.getContent()
            self._proContent
            for root, dirs, files in os.walk(self._startingPath+'/'+self._ProjectName):
                for name in files:
                    filePathList.append(os.path.join(root, name))
            filePathList = [p_ for p_ in filePathList if '.in' in p_]
            filePathList = [f_.split('/')[-2:] for f_ in filePathList]
            filePathList = [f_ for f_ in filePathList if f_[0] in self._proContent]
            return ['/'.join(f_) for f_ in filePathList]
    
        
    
        def runJob(self):
            '''Try QProcess from Qt'''
            dialog = self._dialogD
            self._dialogD._model = QtGui.QStandardItemModel(self._dialogD.uiRunList)
            self._dialogD.uiRunList.setModel(self._dialogD._model)
            self._dialogD.uiRunList.setEditTriggers(QtGui.QAbstractItemView.NoEditTriggers)
            inputList = self.getRunInput()
    
            for calc in inputList:
                item = QtGui.QStandardItem(calc)
                self._dialogD._model.appendRow(item)
    
            self._dialogD.uiRunList.clicked.connect(self.on_item_select2)
            dialog.exec_()
            
    
        def on_item_select2(self, index):
            inputFile = self._startingPath+'/'+self._ProjectName+'/'+str(index.data().toString())
    
            self.process = QtCore.QProcess()
            self.process.setProcessChannelMode(QtCore.QProcess.MergedChannels)
            self.process.readyReadStandardOutput.connect(self.mctdhOut)
            self.results(inputFile)
    
        def results(self, inputFile):
            os.chdir('../Results')
            self._WidgetA.genereInput(inputFile)
            try:
                os.mkdir(self._WidgetA._mainfolder)
            except OSError:
                pass
            mctdh = '/home/piet/newRepo/QuantumDynamics/build/bin/mctdh'
            self.process.start(mctdh+' '+inputFile)
            os.chdir(self._startingPath)
            self._dialogD.close()
    
        def mctdhOut(self):
            output = str(self.process.readAllStandardOutput())
            print output
    
    if __name__ == '__main__':
    
        app = QtGui.QApplication(sys.argv)
        app.setStyle("cleanlooks")
        wnd =Main()
        wnd.show()
        sys.exit(app.exec_())
\end{verbatim}

ModelTree.py:

\begin{verbatim}
    import mctdh
    import sys
    
    class ModelTree(object):
        def __init__(self, config_file='mctdh.config', sys_file='CH3g1.txt'):
            self.config_file = config_file
            self.sys_file = sys_file
            self.bottom_list = []
            self.mode_list =[]
            self.layer_list = []
            self.layer_matr = []
            self.lay_matr_mode = []
            self.label_mode = {}
            self.nodes_spf = {}
    
            print self.config_file, 'from ModelTree'
            print self.sys_file, 'from ModelTree'
            self.config = mctdh.controlParameters()
            self.config.initialize(self.config_file)
            self.basis = mctdh.MctdhBasis()
            self.basis.initialize(self.sys_file, self.config)
            self.node = mctdh.MctdhNode()
            self.phys = mctdh.PhysCoor()
            self.tdim = mctdh.Tdim()
    
            self.getLayerMatr()
            self.getPhysCoord()
            self.modeToGetLayer()
            self.get_SPFs()
    
    
        def getBottomlayer(self):
            """Get the bottom nodes"""
            for i in range(self.basis.NmctdhNodes()):
                self.node = self.basis.MCTDHnode(i)
                if self.node.Bottomlayer() == True:
                    self.bottom_list.append(i)
            return self.bottom_list #List of the i-th bottom node
    
        def getPhysCoord(self):
            """get the Modes of the pys. Coordinates"""
            for i in range(self.basis.NmctdhNodes()):
                self.node = self.basis.MCTDHnode(i)
                if self.node.Bottomlayer() == True:
                    self.phys = self.node.phys_coor()
                    self.mode_list.append(self.phys.mode()) #append Modes to list
    
        def nlayer(self, i):
            """Recursion function that goes from the i-th node up to the top layer of the tree"""
            self.node = self.basis.MCTDHnode(i) #i-th Node
            self.layer_list.append(i)
            if self.node.Toplayer() == False:
                return self.nlayer(self.node.up().address())
            new_list = list(self.layer_list) #copy instead of reference
            del self.layer_list[:]
            return new_list
    
        def getLayerMatr(self):
            """Takes all bottom nodes and returns for each a list of the path to the top node"""
            self.layer_matr = [list(reversed(self.nlayer(b_))) for b_ in self.getBottomlayer()]
    
        def modeToGetLayer(self):
            """Combines layer_matr with mode_list"""
            self.lay_matr_mode = [l_ + [100 + i] for i,l_ in enumerate(self.layer_matr)]
            self.lay_matr_mode = [["Top"] + l_ for l_ in self.lay_matr_mode]
            #concatinates two lists
            self.label_mode = [100 + i for i in range(len(self.mode_list))]
            self.label_mode = dict(zip(self.label_mode, self.mode_list))
    
        def get_SPFs(self):
            """get the SPFs of each Node"""
            for i in range(self.basis.NmctdhNodes()):
                self.node = self.basis.MCTDHnode(i)
                self.tdim = self.node.t_dim()
                self.nodes_spf[i] = self.tdim.GetnTensor() #dict
    
            mode_spf = [self.basis.MCTDHnode(i).t_dim().active(0) for i in \
                        range(self.basis.NmctdhNodes()) if \
                        self.basis.MCTDHnode(i).Bottomlayer() == True ]
    
            mode_spf_dict = dict(zip(self.label_mode, mode_spf))
            for k, i in zip(self.label_mode.keys(), mode_spf):
                self.nodes_spf[k] = i
\end{verbatim}

Node.py:

\begin{verbatim}
    from LogicalNodes import LogicalNodes
    from ModelTree import ModelTree
    import numpy as np
    import networkx as nx
    import re, sys, os
    
    class Parameters(object):
        def __init__(self):
            self._eps_general =  None
            self._eps_1 = None
            self._eps_2 = None
            self._start = None
            self._end = None
            self._dt = None
            self._iteration = None
            self._hamiltonian = None
            self._potential = None
            self._job = None
            self._parameters = None
            self._treeData = None
    
    class InPut(Parameters):
        def __init__(self, filename='example.in'):
            super(InPut, self).__init__()
            self._filename = filename
            self._filenameOld = None
            self._paradict = {}
            self._paralist = []
            self._treelist = []
            self._treeString = ''
            self._commDict = {}
            self.rmCommen()
            self.readFile()
    
        def readFile(self):
            self.getPara("mainfolder")
            self.getPara("Hamiltonian")
            self.getPara("Potential")
            self.getPara("job")
            self.getPara("start")
            self.getPara("end")
            self.getPara("dt")
            self.getPara("iteration")
            self.getPara("out")
            self.getPara2()
            self._paradict['para'] = self._paralist
            self._paradict['Comm'] = self._commDict
            self.getTree()
    
        def file_len(self, fname):
            with open(fname) as f:
                for i, l in enumerate(f):
                    pass
            return i+1
    
        def getPara2(self):
            lineNum = self.file_len(self._filename)
            with open(self._filename, "rb") as text:
                for line in text:
                    if "parameters" in line:
                        for i in range(lineNum):
                            try:
                                para = text.next()
                                if bool(re.search(r'\d', para)):
                                    para = para.split()     #if para contains numbers
                                    self._paralist.append(para)
                            except StopIteration:
                                pass
            lastLine = [x for x in self._paralist[-1] if x != ']']
            del self._paralist[-1]
            self._paralist.append(lastLine)
    
        def rmCommen(self):
            with open(self._filename, "r") as in_put:
                with open('new_InPut.in', 'wb') as output:
                    i = 0
                    for line in in_put:
                        i += 1
                        if '//' in line:
                            self._commDict[i-1] = line
                        else:
                            output.write(line)
    
            self._filename = 'new_InPut.in'              
    
    
        def getPara(self, para):
            with open(self._filename, "rb") as text:
                for line in text:
                    if para in line:
                        try:
                            pos = line.index('=')     # get Index
                            self._paradict[para] = line[pos+1:].strip()  # removes whitestripes
                        except ValueError:
                            pass
    
        def getTree(self):                
            lineNum = self.file_len(self._filename)
            with open(self._filename, "rb") as text:
                for line in text:
                    if 'tree' in line:
                        for i in range(lineNum):
                            try:
                                tree = text.next()
                                if ']' in tree:
                                    break
                                self._treelist.append(tree)
                            except StopIteration:
                                pass
            self._treeString = ''.join(self._treelist)
    
    class OutPut(Parameters):
        def __init__(self, tree, paradict, sysFile, filename="InPut.in"):
            self._mainfolder = paradict['mainfolder']
            self._start = paradict['start']
            self._end =paradict['end']
            self._dt = paradict['dt']
            self._out = paradict['out']
            self._iteration = paradict['iteration']
            self._hamiltonian = paradict['Hamiltonian']
            try:
                self._potential = paradict['Potential']
            except KeyError:
                self._potential = 'no Potential'
            self._job = paradict['job']
            self._parameters = paradict['para']
            self._formated = self.formatparameter()
            self._treeData = tree._treeData
            self._filename = filename
            self._sysFile = sysFile
    
        def savefile(self):
            with open(self._filename, "w") as text_file:
                text_file.write("{0}".format(self.bringAllTogether()))
    
        def savefile2(self):
            with open(self._sysFile, "w") as text_file:
                text_file.write("{0}".format(self.bringTreePara()))
    
        def formatparameter(self):
            output = ""
            A = self._parameters        
            output = '\n'.join(['    '.join(['{:4}'.format(item) for item in row]) for row in A])
            return output
    
        def bringAllTogether(self):
    
            output = "mainfolder = " + self._mainfolder + "\n" \
            "Hamiltonian = " + self._hamiltonian + "\n" \
            "Potential = " + self._potential + "\n" \
            "\n" \
            "job = " + self._job + "\n" \
            "\n" \
            "integrator = { \n" \
            "start = " + self._start + "\n" \
            "end = " + self._end + "\n" \
            "dt = " + self._dt + "\n" \
            "iteration = " + self._iteration + "\n" \
            "out = " + self._out + "\n" \
            "} \n" \
            "\n" \
            "basis = \n" \
            "{\n" \
            "tree = [ \n" \
             + self._treeData + \
            "]\n" \
            "\n" \
            "\n" \
            "parameters = [\n" \
            + self._formated + \
            "]\n" \
            "}"
            return output
    
        def bringTreePara(self):
            output =  self._treeData + \
            "\n" \
            + self._formated
            return output
    
        def __repr__(self):
            return self.bringAllTogether()
    
    class OutPut2(object):
        def __init__(self, paradict, treeString, filename, pathTMP):
            self._parameters = paradict['para']
            self._formated = self.formatparameter()
            self._treeString = treeString
            self._sysFile = pathTMP + '/Load.txt'
            self._TMPmctdhConfig = pathTMP + '/mctdh.config'
            self._path = pathTMP
    
        def savefile(self):
            with open(self._TMPmctdhConfig, "w") as text_file:
                text_file.write("{0}".format(self.bringAllEPS()))
    
        def savefile2(self):
            with open(self._sysFile, "w") as text_file:
                text_file.write("{0}".format(self.bringTreePara()))
    
        def bringAllEPS(self):
            epsList = ['1E-6', '8E-5', '5E-5', '5E-5', '0', '0']
            output = '\n'.join(epsList)
            return output
    
        def bringTreePara(self):
            output =  self._treeString + \
            "\n" \
            + self._formated
            return output
    
        def formatparameter(self):
            output = ""
            A = self._parameters        
            output = '\n'.join(['    '.join(['{:4}'.format(item) for item in row]) for row in A])
    
            return output
    
    
    class Tree(object):
        def __init__(self, mctdhConfig, sys_file):
            self._rootNode0 = Node("TOP")
            self._dictNodes = {}
    
            model = ModelTree(mctdhConfig, sys_file)
            logical = LogicalNodes(model.lay_matr_mode, mctdhConfig, sys_file)
            self._G = logical.G
            self._elder = None
            self.getElder()
            self.addNode(self._elder, str(self._G.nodes[self._elder]['SPF']), self._rootNode0)
            self.readTree()
    
            self._rootNode = self._dictNodes[self._elder]
            self._treeData = self._dictNodes[self._elder].log()
    
        def getElder(self):
            for ele_ in self._G.nodes():
                if self._G.pred[ele_] == {}:
                    self._elder = ele_
            self._elder = self._G.successors(self._elder).next()
    
    
        def readTree(self):
            for suc_ in nx.bfs_successors(self._G, self._elder):
                for brothers in suc_[1]:
                    if 'Mode' in self._G.nodes[brothers].keys():
                        self.addBottomNode(brothers, str(self._G.nodes[brothers]['SPF']), self._dictNodes[suc_[0]], str(self._G.nodes[brothers]['Mode']))
                    else:
                        self.addNode(brothers, str(self._G.nodes[brothers]['SPF']), self._dictNodes[suc_[0]]) #Label, SPF, parent_obj
    
        def setRootNode(self, rootNode):
            self._rootNode = rootNode
            self.setLog()
    
        def setLog(self):
            self._treeData = self._rootNode.log()
    
        def addNode(self, obj, SPF, parent):
            self._dictNodes[obj] = Node(SPF, parent)
    
        def addBottomNode(self, obj, SPF, parent, physcoor):
            self._dictNodes[obj] = BottomNode(SPF, parent, physcoor)





##############################################################    
#    Title: PyQt4 Model View Tutorial Part 04
#    Author: Yasin Uludag
#    Date: 2011
#    Availability: https://www.youtube.com/watch?v=pr1M3mP7qfI
##############################################################

    class Node(object):
        def __init__(self, name, parent=None):
            self._name = name
            self._children = []
            self._parent = parent
    
            if parent is not None:
                parent.addChild(self)
    
        def addChild(self, child):
            self._children.append(child)
    
        def insertChild(self, position, child):
            if position < 0 or position > len(self._children):
                return False
    
            self._children.insert(position, child)
            child._parent = self
            return True
    
        def removeChild(self, position):
            if position < 0 or position > len(self._children):
                return False
    
            try:
                child = self._children.pop(position)
                child._parent = None
                return True
            except IndexError as e:
                print e.message, ': from Node, 237'
                return False
    
        def name(self):
            return self._name
    
        def setName(self, name):
            self._name = name
    
        def child(self, row):
            return self._children[row]
    
        def childAll(self):
            return self._children
    
        def childcount(self):
            return len(self._children)
    
        def parent(self):
            return self._parent
    
        def row(self):
            if self._parent is not None:
                return self._parent._children.index(self)
    
        def log(self, tabLevel=-1):
            output = ""
            tabLevel += 1
    
            for i in range(tabLevel):
                output += "   "
            if self.childcount() == 0:
                output += self._name + "  " + str(self.childcount()) + "\n"
            else:
                output += self._name + " -" + str(self.childcount()) + "\n"
    
            for child in self._children:
                output += child.log(tabLevel)
    
            tabLevel -= 1
    
            return output
    
        def __repr__(self):
            return self.log()
    
        def typeInfo(self):
            return "NODE"
    
    class BottomNode(Node):
        def __init__(self, name, parent, physcoor):
            super(BottomNode, self).__init__(name, parent)
            self._physcoor = physcoor
    
        def typeInfo(self):
            return "Bottom"
    
        def physcoor(self):
            return self._physcoor
    
        def setPhyscoor(self, physcoor):
            self._physcoor = physcoor
    
        def log(self, tabLevel=-1):
            output = ""
            tabLevel += 1
    
            for i in range(tabLevel):
                output += "   "
            if self.childcount() == 0:
                output += self._name + "  " + str(self.childcount()) + "   " + self._physcoor + "\n"
            else:
                output += self._name + " -" + str(self.childcount()) + "\n"
    
            for child in self._children:
                output += child.log(tabLevel)
    
            tabLevel -= 1
    
            return output
\end{verbatim}

View.py:

\begin{verbatim}
    import networkx as nx
    from networkx.drawing.nx_agraph import write_dot, graphviz_layout
    import matplotlib.pyplot as plt
    
    import sys
    from PyQt4 import QtGui
    
    
    class View(object):
        def __init__(self, label_mode, nodes_spf):
            self.label_mode = label_mode
            self.nodes_spf = nodes_spf
            self._G = nx.DiGraph()
        def Display(self, G):
            top = ['Top']
            rest_nodes = [l_ for l_ in self.nodes_spf.keys() if l_ not in self.label_mode.keys()]
            pos = graphviz_layout(G, prog='dot')
            nx.draw(G, pos, with_labels=False, arrows=False, node_color='w')
            nx.draw_networkx_nodes(G, pos, nodelist=top, node_color='w', alpha=1)
            nx.draw_networkx_nodes(G, pos, nodelist=rest_nodes, node_color='r', alpha=1)
    
            pos_lower = {}
            x_off = 0
            y_off = -5
            for k, v in pos.items():
                pos_lower[k] = (v[0] + x_off, v[1] + y_off)
    
            nx.draw_networkx_labels(G, pos_lower, self.label_mode, font_size=16)
    
            pos_higher = {}
            x_off = -7
            y_off = 27
            for k, v in pos.items():
                pos_higher[k] = (v[0] + x_off, v[1] + y_off)
    
            nx.draw_networkx_labels(G, pos_higher, self.nodes_spf, font_size=12)
            plt.savefig('nx_test.png')
            plt.clf()
    
            for key, ele_ in self.nodes_spf.items():
                G.nodes[key]["SPF"] = ele_
            for key, ele_ in self.label_mode.items():
                G.nodes[key]["Mode"] = ele_
            self._G = G
\end{verbatim}


widgetA.py:

\begin{verbatim}
    from PyQt4 import QtCore, QtGui, uic
    import sys
    from Node import OutPut, OutPut2, Tree
    from Node import InPut
    from InputTree import SceneGraphModel
    from ModelTree import ModelTree
    from LogicalNodes import LogicalNodes
    from View import View
    import os, shutil
    
    base, form = uic.loadUiType("dialogA.ui")
    
    class WidgetA(base, form):
        def __init__(self, parent=None):
            super(WidgetA, self).__init__(parent)
            self.setupUi(self)
    
            self._HamiltonianDir = None
            
            ########Attributes######
            self._paradict = {}
            self._integrator = []
            self._tree = None
            self._treeFromLoad = None
    
    
            self._dictHamil = {'CH3Quasie_exact': '1', 'CH4_rst': '2', 'NOCl': '9'}
            self._dictPES = {'CH3Potential': '1', 'PES_CH4': '2', 'PES_HCH4_Zang': '4'}
            self._potential = 'no Potential'
    
            self._mctdhConfig = None 
            self._sysTreeFile = None
            self._inputFile = None
            self._SESmctdhConfig = None
            self._SESsysTreeFile = None
            self._SESinputFile = None
            self._TMPmctdhConfig = None
            self._TMPsysTreeFile = None
            self._TMPinputFile = None
            self._dest = None
    
            self._startingPath = None
            self._ProjectName = None
            self._SessionName = None
            self._temporarySES = None
            self._messagebut = None
    
            #####ListModelHamilton#######
            self._model = QtGui.QStandardItemModel(self.listHamilton)
            for key in self._dictHamil:
                item = QtGui.QStandardItem(key)
                self._model.appendRow(item)
            self.listHamilton.setModel(self._model)
            self.listHamilton.setEditTriggers(QtGui.QAbstractItemView.NoEditTriggers)
            self.listHamilton.clicked.connect(self.on_item_select1)
    
            #####RadioButtonsPES#####
            self.onRadio.setChecked(True)
            self.onRadio.toggled.connect(self.setPES)
            self.offRadio.toggled.connect(self.unsetPES)
            self.offRadio.toggled.connect(self.noPotenial)
    
            #####ListView of PES is built###
            self.setPES()
    
            ###RadioButtonsJob####
            self.RealRadio.toggled.connect(self.setJob1)
            self.RealRadio.setChecked(False)
            self.ImaginaryRadio.toggled.connect(self.setJob2)
            self.EigenstateRadio.toggled.connect(self.setJob3)
            self.fluxEigenstateRadio.toggled.connect(self.setJob4)
    
            self._job = None
            self._dictJob = {'integrate': self.RealRadio,
            'integrate': self.ImaginaryRadio,
            'eigenstates': self.EigenstateRadio,
            'flux eigenstates': self.fluxEigenstateRadio}
    
            ####PushBottoms#####
            self.uiCancel.clicked.connect(self.cancel)
            self.uiSaveJob.clicked.connect(self.saveProject)
            self.uiLoad.clicked.connect(self.FromLoadToTMP)
            self.uiStartCal.clicked.connect(self.runJob)
    
            ####Line Edits#####
            self.uiStartTime.textChanged.connect(self.change1)
            self.uiEndTime.textChanged.connect(self.change2)
            self.uiInit.textChanged.connect(self.change3)
            self.uiIter.textChanged.connect(self.change4)
    
            ####Networkx and MCTDH####
            self.setConfig = None
            self.setSystem = None
            self.modelTree = None
            self.scene = None
    
        def noPotenial(self):
            self._potential = 'no Potential'
    
        def genereInput(self, inputFile):
            ####Get all Parameters from InPut.in#####
            inobj = InPut(inputFile) 
            paradict = inobj._paradict
            self._treeFromLoad = inobj._treeString
            self._integrator = []
            try:
                self._integrator.append(paradict['start'])
                self._integrator.append(paradict['end'])
                self._integrator.append(paradict['dt'])
                self._integrator.append(paradict['iteration'])
                self._integrator.append(paradict['out'])
                self._mainfolder = paradict['mainfolder']
                self._hamiltonian = paradict['Hamiltonian']
                self._job = paradict['job']
                self._parameters = paradict['para']
                self._Comm = paradict['Comm']
                self._potential = paradict['Potential']
            except KeyError as e:
                pass
            if 'no Potential' in self._potential:
            #####RadioButtonsPES#####
                self.offRadio.setChecked(True)
                self.noPotenial()
            ###LineEdit####
            self.uiStartTime.setText(self._integrator[0])
            self.uiEndTime.setText(self._integrator[1])
            self.uiInit.setText(self._integrator[2])
            self.uiIter.setText(self._integrator[3])
    
            self._dictJob[self._job].setChecked(True)
    
        def getInput(self, key):
            ###Files for default Hamiltonians#######
            
            path = self._HamiltonianDir + '/' + key 
            DotIn = self.Finder(path, 'in')
            self._inputFile = path + '/' + DotIn
            
            path = self._HamiltonianDir+'/'+str(key)
    
            sysTreeFile = self.Finder(path, 'txt')
        
            self._mctdhConfig = self._HamiltonianDir + '/' + key + '/' + 'mctdh.config'
            self._sysTreeFile  = self._HamiltonianDir + '/' + key + '/' + sysTreeFile
    
            ###Files for SES calculations####
            if self._ProjectName != None:
                if self._SessionName != None:
                    self._SESmctdhConfig = self._startingPath + '/' + self._ProjectName +'/' + self._SessionName + '/' + 'mctdh.config'
                    self._SESsysTreeFile  = self._startingPath + '/' + self._ProjectName + '/' + self._SessionName + '/' + sysTreeFile
                    self._SESinputFile = self._startingPath + '/' + self._ProjectName + '/' + self._SessionName + '/' + DotIn
    
        def editSession(self, name):
            self.uiProjectName.blockSignals(True)
            self.uiProjectName.setText(str(name))
            self.uiProjectName.blockSignals(False)
    
        def clearSession(self):
            self.uiProjectName.clear()
    
        def makeParaDict(self):
            self._paradict['mainfolder']  = str(self._SessionName)+'/'
            self._paradict['start']       = self._integrator[0]
            self._paradict['end']         = self._integrator[1]
            self._paradict['dt']          = self._integrator[2]
            self._paradict['iteration']   = self._integrator[3]
            self._paradict['out']         = self._integrator[4]
            self._paradict['Hamiltonian'] = self._hamiltonian
    
            try:
                self._paradict['Potential']   = self._potential
            except AttributeError:
                pass
    
            self._paradict['job']         = self._job
            self._paradict['para']        = self._parameters
            self._paradict['Comm']        = self._Comm
    
    
    
        def closeEvent(self, event):
            os.chdir("../")
            event.accept()
    
        def showdialog3(self, Stringmes):
            msg = QtGui.QMessageBox()
            msg.setIcon(QtGui.QMessageBox.Information)
    
            msg.setText(Stringmes)
            msg.setStandardButtons(QtGui.QMessageBox.Save | QtGui.QMessageBox.Cancel)
    
            msg.buttonClicked.connect(self.msgbtn)
            msg.exec_()
    
        def showdialog2(self, Stringmes):
            msg = QtGui.QMessageBox()
            msg.setIcon(QtGui.QMessageBox.Warning)
    
            msg.setText(Stringmes)
            msg.setStandardButtons(QtGui.QMessageBox.Yes | QtGui.QMessageBox.No)
    
            msg.buttonClicked.connect(self.msgbtn)
            msg.exec_()
    
        def msgbtn(self, i):
            self._messagebut = str(i.text())
    
        def showdialog(self, stringMes):
            msg = QtGui.QMessageBox()
            msg.setIcon(QtGui.QMessageBox.Information)
    
            msg.setText(stringMes)
    
            msg.exec_()   
    
        def managefolder(self):
    
            if self._inputFile != None or self._sysTreeFile != None or self._mctdhConfig != None:
    
                shutil.copy2(self._mctdhConfig, self._TMPmctdhConfig)
                shutil.copy2(self._sysTreeFile, self._TMPsysTreeFile)
                shutil.copy2(self._inputFile, self._TMPinputFile)
            else:
                pass
    
        def changeNode(self, my_index):
                topNode = self.modelTree.getNode2(my_index).child(0) #modelTree from SceneGraphModel
                self._tree.setRootNode(topNode)
                
                self.PicGenerate()
    
        def PicGenerate(self):
            ####Generate Outputfiles for new Pic###
            self.output()
    
            ####Pic with MCTDH Code and Networkx####
            if os.path.exists(self._TMPmctdhConfig):
                self.ModelTree = ModelTree(self._TMPmctdhConfig, self._TMPsysTreeFile)
            else:
                print 'Error'
            self.LogicalNodes = LogicalNodes(self.ModelTree.lay_matr_mode, self._TMPmctdhConfig, self._TMPsysTreeFile) #object
            self.View = View(self.ModelTree.label_mode, self.ModelTree.nodes_spf) #object
            self.View.Display(self.LogicalNodes.G) #View method Display() generated .png file
    
            ####QGraphicsView###
            pixmap = QtGui.QPixmap('nx_test.png')
            self.scene = QtGui.QGraphicsScene(self)
            self.scene.addPixmap(pixmap)
            self.uiDisplayTree.setScene(self.scene)
    
        def New_Session(self):
            name = str(self.uiProjectName.text())
            if self._SessionName == None:
                    print name
    
        def SESfiles(self):
            
            path = self._startingPath + '/' + self._ProjectName + '/tmp'
            DotIn = self.Finder(path, 'in')
            sysTreeFile = self.Finder(path, 'txt')
    
            if self._ProjectName != None:
                if self._SessionName != None:
                    self._SESmctdhConfig = self._startingPath + '/' + self._ProjectName +'/' + self._SessionName + '/' + 'mctdh.config'
                    self._SESsysTreeFile  = self._startingPath + '/' + self._ProjectName + '/' + self._SessionName + '/' + sysTreeFile
                    self._SESinputFile = self._startingPath + '/' + self._ProjectName + '/' + self._SessionName + '/' + DotIn
    
        def saveProject(self):
            name = str(self.uiProjectName.text())
            self._SessionName = name
            self.SESfiles()
            Profiles = os.walk(self._startingPath+'/'+self._ProjectName+'/').next()[1]
            if name in Profiles:
                SESfiles = os.walk(self._startingPath+'/'+self._ProjectName+'/'+name).next()[2]
    
                ###Checks if SES contains files###        
                if SESfiles:
                    self.showdialog2('Overwriting %s?' %name)
                    if 'Yes' in self._messagebut:
                        self.output()  
                        self.fromTMPToSES()
                        ###Saves all Parameter and Tree to *.in and tree only to *.txt### 
                        self.esc()
                    else:
                        pass
                else:
                    TMPfiles = os.walk(self._startingPath+'/'+self._ProjectName+'/tmp').next()[2]
                    if TMPfiles:
                        self.output()
                        self.fromTMPToSES()
                        self.esc()
                    else:
                        self.showdialog('Nothing to save?')
    
            else:
                if name == '':
                    print name
                    self.showdialog('Please give Session name')
                else:
                    os.chdir(self._startingPath+'/'+self._ProjectName)
                    os.mkdir(name)
                    os.chdir(self._startingPath)
                    self.output()
                    self.fromTMPToSES()
                    self.esc()
            if self._potential == 'no Potential':
                with open(self._SESinputFile, 'r') as f:
                    lines = f.readlines()
                with open(self._SESinputFile, 'w') as f:
                    for line in lines:
                        if 'Potential' not in line:
                            f.write(line)
                
        def cancel(self):
            self.removeContent()
            self.esc()
    
        def removeContent(self):
            TMPpath = self._startingPath +'/'+ self._ProjectName + '/tmp'
    
            try:
                shutil.rmtree(TMPpath)
            except OSError:
                raise
                
            sysPath = self._startingPath +'/'+ self._ProjectName
            os.chdir(sysPath)
            os.mkdir('tmp')
            os.chdir(self._startingPath)
    
        def copyLoad(self):
            LOADinputFile = str(QtGui.QFileDialog.getOpenFileName())
            try:
                shutil.copy2(LOADinputFile, self._TMPinputFile)
            except Exception:
                raise
    
    
        def FromLoadToTMP(self):
            
            self.clearTree()
            
            self._TMPinputFile =  self._startingPath  + '/' \
            + self._ProjectName + \
            '/tmp/InPut.in'
            
            self._SESsysTreeFile =  self._startingPath  + '/' \
            + self._ProjectName + '/' \
            + self._SessionName + '/Load.txt'
    
            self._SESinputFile =  self._startingPath  + '/' \
            + self._ProjectName + '/' \
            + self._SessionName + '/InPut.in'
    
            ###removes tmp folder's content###
            self.removeContent()
    
            ###copies *.in file to tmp folder###
            self.copyLoad()
    
            ###generates Parameter from *.in file###
            self.genereInput(self._TMPinputFile)
            self.makeParaDict()
            pathTMP = self._startingPath + '/' + self._ProjectName + '/tmp'
            outobj = OutPut2(self._paradict, self._treeFromLoad, self._TMPinputFile, pathTMP)
            outobj.savefile()
            outobj.savefile2()
            ###Tree will be constructed from parameters###
            self._TMPmctdhConfig = self._startingPath + '/' \
            + self._ProjectName + \
            '/tmp/Load.txt'
    
            self._TMPsysTreeFile = self._startingPath + '/' \
            + self._ProjectName + \
            '/tmp/mctdh.config'
    
            self.TreeOnly()
                
    
        def checkTMP(self):
            sysPathTMP = self._startingPath +'/'+ self._ProjectName +'/tmp'
            files = os.walk(sysPathTMP).next()[2]
            if files:
                return True
            return False
    
        def fromHToTMP(self, item):
            if self.checkTMP():
    
                self.showdialog2('Overwriting temporary Settings?')
                if 'Yes' in self._messagebut:
                    self.fromHToTMPinner(item)
                else:
                    pass
            else:
                self.fromHToTMPinner(item)
    
        def fromHToTMPinner(self, item):
            self.clearTree()
            sysPath = self._HamiltonianDir+'/'+item
    
            sysFile = self.Finder(sysPath, '.txt')
            DotIn = self.Finder(sysPath, '.in')
    
            self._mctdhConfig = sysPath+'/'+'mctdh.config'
            self._sysTreeFile = sysPath+'/'+sysFile
            self._inputFile   = sysPath+'/'+DotIn
    
            self._TMPmctdhConfig = self._startingPath + '/' \
            + self._ProjectName + \
            '/tmp/mctdh.config'
    
            self._TMPsysTreeFile = self._startingPath + '/' \
            + self._ProjectName + \
            '/tmp/' + sysFile
    
            self._TMPinputFile =  self._startingPath  + '/' \
            + self._ProjectName + \
            '/tmp/' + DotIn
    
    
            try:
                shutil.copy2(self._mctdhConfig, self._TMPmctdhConfig) 
                shutil.copy2(self._sysTreeFile, self._TMPsysTreeFile)
                shutil.copy2(self._inputFile, self._TMPinputFile)
            except Exception:
                raise
    
        def fromSESToTMP(self, sysFile):    
    
            self._TMPmctdhConfig = self._startingPath + '/' \
            + self._ProjectName + \
            '/tmp/mctdh.config'
    
            self._TMPsysTreeFile = self._startingPath + '/' \
            + self._ProjectName + \
            '/tmp/' + sysFile
    
            SESpath =  self._startingPath  + '/' \
            + self._ProjectName+ '/'  +\
            self._SessionName
    
            DotIn = self.Finder(SESpath, '.in')
    
            self._TMPinputFile =  self._startingPath  + '/' \
            + self._ProjectName  +\
            '/tmp/' + DotIn
    
            try:
                shutil.copy2(self._SESmctdhConfig, self._TMPmctdhConfig) 
                shutil.copy2(self._SESsysTreeFile, self._TMPsysTreeFile)
                shutil.copy2(self._SESinputFile, self._TMPinputFile)
            except Exception:
                raise
    
        def fromTMPToSES(self):    
            TMPpath = self._startingPath + '/' \
            + self._ProjectName + '/' + \
            'tmp'
    
            sysFile = self.Finder(TMPpath,'txt')
            DotIn = self.Finder(TMPpath,'.in')
    
            self._TMPmctdhConfig = self._startingPath + '/' \
            + self._ProjectName + \
            '/tmp/mctdh.config'
    
            self._TMPsysTreeFile = self._startingPath + '/' \
            + self._ProjectName + \
            '/tmp/' + sysFile
    
            self._TMPinputFile =  self._startingPath  + '/' \
            + self._ProjectName +\
            '/tmp/' + DotIn
    
            try:
                shutil.copy2(self._TMPmctdhConfig, self._SESmctdhConfig) 
                shutil.copy2(self._TMPsysTreeFile, self._SESsysTreeFile)
                shutil.copy2(self._TMPinputFile, self._SESinputFile)
            except Exception:
                raise
            
    
        def esc(self):
            self.close()
    
        def setSessionName(self, name):
            self._SessionName = name
    
        def folderExist(self):
            folders = os.walk(self._startingPath+'/'+self._ProjectName).next()[1]
            if self._SessionName in folders:
                self.showdialog('Folder already exists!')
                return False
            return True
            
        def change0(self):
            self._SessionName = str(self.uiProjectName.text())
    
        def change1(self):
            self._integrator[0] = str(self.uiStartTime.text())
        def change2(self):
            self._integrator[1] = str(self.uiEndTime.text())
        def change3(self):
            self._integrator[2] = str(self.uiInit.text())
        def change4(self):
            self._integrator[3] = str(self.uiIter.text())
    
        def setJob1(self):
            self._job = "integrate"
        def setJob2(self):
            self._job = "integrate"
        def setJob3(self):
            self._job = "eigenstates"
        def setJob4(self):
            self._job = "thermalflux"
    
        def unsetPES(self):
            self.modelPES.removeRows(0, len(self._dictPES), QtCore.QModelIndex())
    
        def setPES(self):
            #####ListModelPES#######
            self.modelPES = QtGui.QStandardItemModel(self.listPES)
            for key in self._dictPES:
                item = QtGui.QStandardItem(key)
                self.modelPES.appendRow(item)
            self.listPES.setModel(self.modelPES)
            self.listPES.setEditTriggers(QtGui.QAbstractItemView.NoEditTriggers)
            self.listPES.clicked.connect(self.on_item_select2)
    
        def makedir(self):
            path = self._startingPath + '/' + self._ProjectName + '/tmp'
            try:
                os.makedirs(path)
            except (IOError, OSError) as e:
                pass
    
        def start(self):
            self.clearTree()
            name = str(self.uiProjectName.text())
            self._SessionName = name
            #### if SES contains files, these files will be copied to TMP
            try:
                filenames = os.walk(self._startingPath+'/'+self._ProjectName+'/'+self._SessionName).next()[2]
                for val in filenames:
                    if 'txt' in val:
                        sysTreeFile = val
    
                self._SESmctdhConfig = self._startingPath + '/' + self._ProjectName +'/' + self._SessionName + '/' + 'mctdh.config'
                self._SESsysTreeFile  = self._startingPath + '/' + self._ProjectName + '/' + self._SessionName + '/' + sysTreeFile
                self._SESinputFile = self._startingPath + '/' + self._ProjectName + '/' + self._SessionName + '/' + 'InPut.in'
    
                ###copies files from SES to TMP
                self.fromSESToTMP(sysTreeFile)
    
                ###Parameters from InPut.in in TMP will be loaded####
                self.genereInput(self._TMPinputFile)
    
                ###Tree will be constructed from parameters###
                self.TreeOnly()
    
            except (StopIteration, UnboundLocalError):
                ###when UnboundLocalError is raised then there are no files in SES###
                # -> deleteing tmp
                self.removeContent()
                self.clearTree()
        
        def clearTree(self):
                    try:
                        self.scene.clear()
                        self.uiDisplayTree.setScene(self.scene)
                        self.modelTree.removeRow(0)
                    except (IndexError, AttributeError) as e:
                        pass
    
        def Finder(self, path, app):
            fileList = os.walk(path).next()[2]
            return [f_ for f_ in fileList if app in f_][0]
    
        def TreeOnly(self):
            ####TreeView########
    
            TMPpath = self._startingPath+'/'+self._ProjectName+'/tmp'
            self._TMPmctdhConfig = TMPpath+'/mctdh.config'  
            textFile  = self.Finder(TMPpath, 'txt')
            self._TMPsysTreeFile = TMPpath+'/'+textFile 
            self._tree = Tree(self._TMPmctdhConfig, self._TMPsysTreeFile)
            self.modelTree = SceneGraphModel(self._tree._rootNode0)
            self.uiTree.setModel(self.modelTree)
            self.uiTree.expandAll()
            self.uiTree.resizeColumnToContents(0)
            self.uiTree.resizeColumnToContents(1)
            self.uiTree.clicked.connect(self.changeNode)
            #####make Pic from tmp###
            self.PicGenerate()
    
        def generateTree(self, item):
            key = item
            self.getInput(key)
            
            self._hamiltonian = self._dictHamil[str(key)]
    
    
            #####generates Tree###
            self.TreeOnly()
    
    
        def on_item_select1(self, item):
    
            key = str(item.data().toString())        
    
            ####Copy from default Hamilton to tmp
            self.fromHToTMP(key)
    
            ####Generate input from *.in to self._paradict####
            self.genereInput(self._TMPinputFile)
    
            ####Building Tree####
            self.generateTree(key)
    
        def on_item_select2(self, item):
            key = item.data().toString()
            self._potential = self._dictPES[str(key)]
    
        def output(self):
            """Class OutPut takes all parameters and saves them in File by creating
         the object of this class"""
            self.makeParaDict()
            outobj = OutPut(self._tree, self._paradict, self._TMPsysTreeFile, self._TMPinputFile)
            outobj.savefile()
            outobj.savefile2()
    
        def runJob(self):
            self.esc()
            self.process = QtCore.QProcess()
            self.process.setProcessChannelMode(QtCore.QProcess.MergedChannels)
            self.process.readyReadStandardOutput.connect(self.mctdhOut)
            inputFile = self._startingPath+'/'+self._ProjectName+'/'+self._SessionName
            DotIn = self.Finder(inputFile, 'in')
            inputFile = inputFile+'/'+DotIn
            self.results(inputFile)
    
        def results(self, inputFile):
    
            os.chdir(self._startingPath+'/../Results')
            self.genereInput(inputFile)
            try:
                os.mkdir(self._mainfolder)
            except OSError:
                pass
            mctdh = '/home/piet/newRepo2/QuantumDynamics/build/bin/mctdh'
            self.process.start(mctdh+' '+inputFile)
            os.chdir(self._startingPath)
    
        def mctdhOut(self):
            output = str(self.process.readAllStandardOutput())
            print output
\end{verbatim}